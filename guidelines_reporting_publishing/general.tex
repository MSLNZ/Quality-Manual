\section{General style guidelines}
\subsection{Numbers and units}
Numbers and units should be formatted following the NIST guide to the SI \cite{NIST_SI}.

Quantity values with units should not be split over lines. Use a non-breaking spacing to prevent this (in MS Word: CTRL-SHIFT-SPACE).

When numbers are expressed with more than four digits (either side of the decimal marker), a thin space is preferred to a coma, or nothing, as separator between groups of three digits. Some examples are shown in figure~\ref{fig:ex_number_formats} 
\begin{figure}[h]
\begin{center}
\begin{tabular}{Sl}
{do this} &  {not this} \\ \hline 
76483522 & 76,483,522 \\
43279.168 & 43,279.168 \\
8012 & 8,012 \\ 
18012 & 18,012 \\
0.4917223 & 0.4917223 
\end{tabular}
\caption{Examples of preferred number formatting. Note also that all numbers are aligned on the location of the decimal point in the left-hand column.}
\label{fig:ex_number_formats}
\end{center}
\end{figure}

\subsubsection{Generating a space separator}
It is not easy to generate a thin space in Microsoft Word. The following recipe is a possibility:
\begin{enumerate}
\item Type a space normally (use a non-breaking space if needs be) 
\item Select the space and right click to obtain the context menu; select Font …
\item Select the Advanced tab and set Spacing to `Condensed' (the ``by 1 pt'' default is OK)
\item Click OK and copy the space to the clipboard 
\end{enumerate}

This thin space character can be pasted into numbers as required.

Using Microsoft Excel it is possible to select a normal width space separator for thousands in the number formatting. Unfortunately, this does not insert spaces between groups of three digits to the right of the decimal marker. 

To set up Excel to insert a space between digits to the left of the decimal marker:
\begin{enumerate}
\item Find the `Use system separators' check box under `Editing Options' in the Excel Options menu 
\item Un-check this box and type a space character into the `Thousands Separator' box
\end{enumerate} 

Alternatively, to insert spaces between groups of three digits to the right of the decimal marker use a custom number format of ``\verb|#.### ### ###|''. Similarly, a custom number format of ``\verb|# ### ###|'' can be used to set a space as the thousands separator.

\subsection{Mathematical language}
Mathematical formulae and equations should be contained in complete sentences with appropriate punctuation. 

Punctuation follows equations: a computation that ends a sentence needs to end with a full stop; a computation that does not end a sentence should be followed by a comma (an effective way to check for appropriate punctuation is to read text containing equations out loud). 

For example, the following sentence appears in a report for a transformer test set: 
\[
\text{``For $\rho = \SI{-1}{\%}$ and $\varphi= \SI{1}{crad}$, }
\varepsilon = \frac{1=0.01}{\cos(0.01)} - 1 \; =\; \SI{-0.995}{\%}\,. 
\text{''}
\]

Often, the use of space and separate lines for equations can be helpful to a reader, as it is in this extract from a calibration report:
\begin{quote}

A comparator is used to evaluate the vector ratio QX/QN where QX is the fundamental component of the signal vector at terminals X (either voltage or current) and QN is the fundamental component of the corresponding signal vector at terminals N. 

This ratio can be represented in polar co-ordinates as:
\[
\frac{Q_X}{Q_N} = (1 + \varepsilon) e^{\mathrm{j}\varphi} \;,
\]
where:

\begin{tabular}{rl}
$\varepsilon$ & is the ratio error, and \\
$\varphi$ & is the phase error, for the polar co-ordinate system. 
\end{tabular}

\vspace{\baselineskip}
This ratio can be also represented in rectangular co-ordinates as:
\[
\frac{Q_X}{Q_N} = 1 + \rho + \mathrm{j}\cdot\tan (\gamma) \;,
\]
where
\begin{tabular}{rl}
	$\rho$ &  is the ratio error, and \\
	$\tan (\gamma)$ & is the phase error, for the rectangular co-ordinate system. 
\end{tabular}

\vspace{\baselineskip}	
This ELTEL comparator reports the ratio error as $\rho$ (rectangular co-ordinate system) and the phase error as $\varphi$ (polar co-ordinate system).
\end{quote}

\subsubsection{Symbols and notation}

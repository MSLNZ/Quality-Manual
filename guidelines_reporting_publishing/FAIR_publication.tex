\proposedbox{
\section{Technical documents published by MSL}
\label{s:MSL_published_documents}

MSL self-publishes Technical Guides and (Callaghan Innovation) Technical Reports. These documents need to be archived in an appropriate scientific repository. 

The preferred repository for MSL publications is Zenodo (\url{https://zenodo.org}), which is operated by CERN and available for use by anyone.  

\subsection{Zenodo}
\label{ss:zenodo}

\subsubsection{Accounts} Individuals need a personal account to use Zenodo. An account can be created using existing ORCID or Github accounts, or directly on the Zenodo platform (see \url{https://help.zenodo.org/docs/get-started/create-an-account/}).

\subsubsection{Using Zenodo}
There is an excellent description of how to use Zenodo on YouTube.
\begin{center}
\url{https://youtu.be/BPVSErzNtME?si=7TNy9OVy2FhrkN6D}.
\end{center}
The platform is easy to use. Note that it is possible to provide the information required in stages, saving the incomplete record until everything is ready to publish. In particular, a DOI can be reserved for use before the document itself is finalised. This allows DOIs for Technical Guides and Technical Reports to be obtained and inserted in the final documents.

The important metadata elements to enter are:
\begin{description}
    \item[Upload type:] a `publication' in the case of technical guides and technical reports;
    \item[DOI:] use the `Reserve DOI' feature to obtain this for technical guides and technical reports;
    \item[Title:] same as the technical guide or report;
    \item[Authors:] follow the suggested format for name, use `Measurement Standards Laboratory of New Zealand' for the affiliation, and enter the author's ORCID;
    \item[Description:] use the document abstract if possible, otherwise create a good description, because this text is visible to people browsing the archive;
    \item[Keywords:] use if possible to enhance find-ability;
    \item[Access rights:] open access for technical guides and technical reports;
    \item[License:] there are many licenses available, but they do not all appear until you start typing into the dialog box  
\end{description}

When everything is ready, remember to push the `Publish' button at the top right of the page.

\subsubsection{MSL Communities} MSL keeps collections of archived publication records on Zenodo, which are called \href{https://help.zenodo.org/docs/communities/about-communities/}{communities}. There are MSL communities for (\textbf{URLs will be available}):
\begin{itemize}
    \item Technical Guides
    \item Technical Reports
\end{itemize}

MSL's communities are owned and curated by the Chief Metrologist (\textbf{to be confirmed}). 

Publication records are always deposited in Zenodo by an individual. 

After deposit, a record should be submitted to a community. The community's curator is responsible for accepting each submission. 

Individual publications (e.g., a particular technical report or technical guide) will each require an `upload' (Zenodo teminology). 

When a publication is uploaded for the first time, the record must be named and a DOI will be issued. A lot of metadata can be provided at this stage. Everything can be saved, and re-edited,  during this process until the submission is \textit{published} on Zenodo. 

After publication, changes cannot be made to the publication itself and the upload cannot be easily removed (deleted). Some changes to the metadata are permitted. 

However, revised versions of an earlier upload can be deposited. This is done by the individual who made the initial upload. After selecting the link to an upload, the option of depositing a revised version becomes available. The new version is issued with a distinct DOI.    

%\subsection{The FAIR principles}
%The FAIR principles can be applied to enhance the value of digital assets, like our publications, in terms of use by others \cite{FAIR}. 
%
%In relation to MSL publications, they can be interpreted as follows:
%\begin{description}
%	\item[Findable: ] A document must have a unique identifier (e.g., for searching and indexing purposes).  A Digital Object Identifier (DOI) is appropriate \cite{doi}. Several public science repositories issue a DOI when a document is deposited. Using one of these also takes care of the `A' principle, because the repository will also ensure that the document remains accessible.
%	
%	Authors should also be identified: each author should include their ORCID in the document \cite{orcid}.
%	
%	\item[Accessible: ] A document must be accessible (online). A distinction is made between knowing that a document exists (e.g., finding it in a search) and accessing it. 
%	
%	A document may still be protected from unauthorised use (e.g., by encryption) so openness is not implied.
%	
%	\item[Interoperable: ] A variety of applications should be able to use of the document. Publishing documents in PDF format (or better, in the PDF-A format) ensures that prospective readers have a choice of reading tools.
%	  
%	\item[Reusable: ] We publish documents for others to read, but sometimes also to reuse or re-distribute. Without a clear statement about copyright and licensing, a reader does not know if we give permission for the material to be reused. Permission should be explicitly granted, with an indication of acceptable ways of doing so. The Creative Commons licence, described in \S\ref{s_copyright}, is suitable. 
%
%The provenance of work needs to be acknowledged too. Advice on how to cite (reference) a work should be included. 
%
%\end{description}  

\subsection{Guidelines}
\subsection{Technical reports}
The figure below shows the appearance of an MSL technical report. Note the hyperlink to the DOI, where copies of the report are available. The green icon after the author's name is the orcid icon and is a hyperlink to orcid.

\begin{center}
\includegraphics[scale=.5,page=1]{pictures/Report}
\end{center}

\newpage
The inside page of an MSL technical report repeats the title and includes a `reference', which suggests how the work can be cited, as shown below. The Creative-Commons licence is automatically placed after the summary / abstract material. 

\begin{center}
\includegraphics[scale=.5,page=2]{pictures/Report}
\end{center}

\subsection{Technical Guides}
The figure below shows the appearance of an MSL Technical Guide. The DOI is given in the header but not as a hyperlink.
\begin{center}
\includegraphics[scale=.5,page=1]{pictures/TG_Template}
\end{center}

\newpage
The last page of Technical Guide has an end-matter section where the names and orcids of the people who prepared the guide appear with other contact details. The doi is repeated here, but this time as a hyperlink. The Creative-Commons licence is automatically included.

\begin{center}
\includegraphics[scale=.5,page=2]{pictures/TG_Template}
\end{center}
}%
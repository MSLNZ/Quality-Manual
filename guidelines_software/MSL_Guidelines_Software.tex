% --------------------------------------------------------------------
% Preamble
% --------------------------------------------------------------------
\documentclass[a4paper]{article}	

\usepackage[scaled]{helvet}
\renewcommand\familydefault{\sfdefault} 
\usepackage[T1]{fontenc}

% 'microtype' can only be used with pdflatex
\usepackage[protrusion=true,expansion=true]{microtype}
	
\usepackage[utf8]{inputenc}
\usepackage[english]{babel}
\usepackage{amsmath,amsfonts,amsthm,amssymb,bm}
\usepackage{graphicx}
\usepackage{xcolor}
\usepackage{enumitem} 	% Allows different styles in enumerate
\usepackage{ulem}		% For strikethrough text
\usepackage{rotating}	% For landscape figures

% Spacing between lines 
\usepackage{setspace}
\onehalfspacing

% ----- page layout 
\usepackage[
	a4paper,
	top=1in, bottom=1.25in, left=1.25in, right=1.25in
]{geometry}	

\setlength{\oddsidemargin}{5mm}			% Remove 'twosided' indentation
\setlength{\evensidemargin}{5mm}

% ----- To allow cross-referencing between documents
%       a common root folder must be established
%       this command encapsulates the path
\newcommand{\gsdPath}{../guidelines_software}%

% ------ To cross reference other documents
\usepackage{xr-hyper}
\usepackage[
	colorlinks=true,
	urlcolor=magenta,
	linkcolor=blue, 	% [red]
%	anchorcolor 	 % [black]
	citecolor=blue, 		% [green]
	filecolor=cyan, 	% [cyan]
	menucolor=black		% [red]
]{hyperref}

% Header and footer control
\usepackage{fancyhdr}
\setlength{\headheight}{15.2pt}
\pagestyle{fancy}

% ----- Table of contents 
\usepackage{titletoc}
\usepackage{tocloft}

\renewcommand\cftsecfont{\normalfont}
\renewcommand\cftsecpagefont{\normalfont}
\renewcommand{\cftsecleader}{\cftdotfill{\cftsecdotsep}}
\renewcommand\cftsecdotsep{\cftdot}
\renewcommand\cftsubsecdotsep{\cftdot}


% ----- Tables
\usepackage{booktabs}
\usepackage{colortbl}
\usepackage{longtable}
\usepackage{makecell}

% ----- Units formatting 
\usepackage{siunitx}

% ----- For software snipets (set up for Python) 
\usepackage{listings}
\lstset{ %
language=Python,                	% the language of the code
basicstyle=\footnotesize\ttfamily,  % the size of the fonts that are used for the code
basewidth=0.51em,
columns=fullflexible,
%stringstyle=\color{white}\ttfamily,
%numbers=left,                   	% where to put the line-numbers
%numberstyle=\tiny,      			% the size of the fonts that are used for the line-numbers
%stepnumber=1,                   	% the step between two line-numbers. If it's 1, each line
                                	% will be numbered
%numbersep=5pt,                  	% how far the line-numbers are from the code
backgroundcolor=\color{yellow!15},  % choose the background color. You must add \usepackage{color}
showspaces=false,               	% show spaces adding particular underscores
showstringspaces=false,         	% underline spaces within strings
showtabs=false,                 	% show tabs within strings adding particular underscores
%frame=single,                   	% adds a frame around the code
tabsize=4,                      	% sets default tabsize to 2 spaces
captionpos=b,                   	% sets the caption-position to bottom
breaklines=true,                	% sets automatic line breaking
breakatwhitespace=false,        	% sets if automatic breaks should only happen at whitespace
title=\lstname,                 	% show the filename of files included with \lstinputlisting;
                                	% also try caption instead of title
%escapeinside={\%*}{*)},         	% if you want to add a comment within your code
escapeinside={(*}{*)},
%morekeywords={*,...}            	% if you want to add more keywords to the set
belowskip=-1.75\baselineskip		% 1.5 seems to be best
}

% ------ Macros for this document
\definecolor{yellowMSL}{RGB}{244,182,8}

% Mark-up for small changes (a few words only)
\newcommand{\proposed}[1]{{\color{cyan}{{#1}}}}
\newcommand{\deprecated}[1]{{\color{yellowMSL}{#1}}}

% Mark-up  for paragraphs. 
\usepackage{framed}
\newcommand{\proposedbox}[1]{
\colorlet{shadecolor}{gray!10}
{\color{cyan}\begin{shaded}
	{#1}
\end{shaded}}
}%

\newcommand{\deprecatedbox}[1]{
\colorlet{shadecolor}{gray!10}
{\color{yellowMSL}\begin{shaded}
	{#1}
\end{shaded}}
}%

% --------------------------------------------------------------------
% Front page definitions (do not change this)
% --------------------------------------------------------------------
\newcommand{\HRule}[1]{\rule{\linewidth}{#1}} 	% Horizontal rule

\makeatletter							% Title
\def\printtitle{%						
    {\centering \@title\par}}
\makeatother									

\makeatletter							% Author
\def\printauthor{%					
    {\raggedright \Large \@author}}				
\makeatother							


% --------------------------------------------------------------------
% Details of what to put on the front page (Change this)
% --------------------------------------------------------------------
\title{	\Large 
	\textsc{ Measurement Standards Laboratory of New Zealand } \\ [2.0cm]			
	\HRule{2pt} \\ [0.25cm]						
	\LARGE \textbf{\uppercase{Software Development Guidelines}}	\\% Title
    \Large \textit{A supplement to the MSL Quality Manual}
	\HRule{2pt} \\ [0.25cm]		
	\Large \today			
}

\author{
{\large 
	Text in this \proposed{colour} is awaiting approval by the 
	Quality Council.\\ 
	Text in this \deprecated{colour} is deprecated 
	and awaiting approval to be deleted.
} \\[\baselineskip]
 		
Measurement Quality Council\\	
Measurement Standards Laboratory of New Zealand\\	
}

\begin{document}
\hypersetup{pageanchor=false}	% prevents a warning message
\fancyhf{}	% Header and footer are empty

% ------------------------------------------------------------------------------
% Title page
% ------------------------------------------------------------------------------
\thispagestyle{empty}		% Remove page numbering on this page
\pagestyle{plain}			% puts pages numbers back in front matter

\printtitle					% Print the title data as defined above
  	\vfill
\printauthor				% Print the author data as defined above
\newpage

\pagenumbering{roman}

\tableofcontents
\newpage

% ----- Main document now begins
\pagestyle{fancy}			% Turn on footer style control
\pagenumbering{arabic}
\hypersetup{pageanchor=true}
\setcounter{page}{1}		% Set page numbering to begin on this page

\fancyfoot[L]{Measurement uncertainty}
\fancyfoot[C]{\today}
\fancyfoot[R]{Page \thepage\ of \pageref*{LastPage}}

\setlength{\parindent}{0cm}	% No indent when starting a new paragraph
\setlength{\parskip}{\baselineskip}	% Leave a blank line between paragraphs

% ----- These files contain the document content
\section{Introduction}
\begin{flushright}
\textit{“If you don’t know where you’re going, any road will take you there.”} \\
-- George Harrison 
\end{flushright}

MSL relies heavily on software to control experiments and to acquire, process and present data. Most technical sections develop bespoke solutions that suit their needs.\footnote{It is becoming increasingly common to use `data' to refer to all forms of digital objects, so software and publication documents are also `data'.} 

Software must meet objective requirements set to accomplish a specific task. This will be verified during validation of a technical procedure. The quality system also requires the integrity of validated software and measurement data to be controlled. 

These are rather modest requirements for assets that represent a very significant investment of MSL's time to develop and maintain.  Perhaps unsurprisingly, problems arising from insufficient maintenance of software are rife. 

A lot of MSL software consists of `legacy' programs that are only occasionally used. Specification of requirements, documentation and software testing are often lacking, making maintenance difficult. Some sections have also invested heavily in types of software that are inherently difficult to maintain at the level required for calibration work (spreadsheets being a good example: convenient for quick work, but notorious for hiding errors and allowing inadvertent changes to creep in). 

Another problem is the platform required to run legacy software. IT systems evolve much faster than MSL would wish, so older hardware, and older software tools (e.g., compliers) need to be operated after they would have retired elsewhere. In many cases, these older systems are retained because not enough is known to allow migration.

There are general practices that can improve the reliability and maintainability of software. This guide recommends a planned and structured approach to software development. The objective is not merely to produce an executable application, it is also to provide: a clear statement of  requirements, an explanation of the design adopted to meet those requirements and a suite of tests that verifies the software's performance against requirements. It is essential that any part of a software project can be examined and understood by someone other than the author.  

We recommend that software development be undertaken in active consultation among members of the technical section, or wider MSL. The quality of one individual's work will benefit from a review by someone else. Professional developers use regular reviews to both detect errors and improve the quality of their work. These reviews occur regularly, not just at the end of a project.   


\clearpage
\section{Style and structure}

\begin{flushright}
\textit{If you don’t know where you’re going, any road will take you there.}\\
--- George Harrison
\end{flushright}

Software should be developed with end-users and future owners in mind; it should also be designed to be tested and modified easily. All of which requires consideration about how software will be structured and written.

\subsection{Style and idiom}
Writing software, writing prose, or even writing mathematical formulae, has much in common.

Any language (English, French, mathematics, etc) has syntactical rules and common idioms. We work with these to create statements that others can readily understand. A programming language is much the same. Although each language has its features and quirks, there will be good ways of describing operations and there will be awful alternatives; there will be `best-practice' and there will be ghastly `hacks'. When writing software, consideration should always be given to another reader (or even the author at some distant point in time): be consistent, express yourself using a simple, clear style.

A worthy goal is to write software that is self-explanatory. 

\paragraph{Names} 
Procedural languages (such as Visual BASIC, Python, C++, etc) and some end-user tools (e.g., MathCAD) allow considerable freedom when naming things. Use this freedom to make software readable. Names should be concise but descriptive: a reader should easily keep track of the meanings of different variables, functions and classes, etc.

\paragraph{Format and layout} The appearance of source code and high-level programming environments, like LabView and spreadsheets, is important. Adopt clear formatting conventions and layout that reflect the structure of a program. 

\paragraph{Comments} Somewhat counter-intuitively, comments in software are really a necessary evil: they are needed, but should be used as sparingly as possible. It is always better to re-phrase the code itself, using appropriate naming, layout and idioms, if doing so will make a comment unnecessary.

There will often be stylistic conventions in a language. Use them, because readers are more likely to understand your code. If there are alternatives styles, choose one and stick to it. Be consistent: don't swap back and forth between one style and another. 

\clearpage

\appendix	% Changes the style of the headings for sections
% \section{Abbreviations}
\begin{center}
{\renewcommand*{\arraystretch}{1.4}
\begin{tabular}{p{14.07em}p{25em}}
	\rowcolor[rgb]{ 0,  0,  0} 
	\textcolor[rgb]{ 1,  1,  1}{\textbf{Abbreviation}} & 
	\textcolor[rgb]{ 1,  1,  1}{\textbf{Stands For}} \\
APMP & Asia Pacific Metrology Programme \\ 
BIPM & Bureau International des Poids et Mesures \\ 
CC & Consultative Committee of the BIPM \\ 
CGPM & Conf\'erence G\'en\'erale des Poids et Mesures \\ 
CIPM & Comit\'e International des Poids et Mesures \\
CMC & Calibration and Measurement Capability \\
DI & Designated Institute \\
EDRMS & Electronic Document and Records Management System \\
EU & European Union \\
IANZ & International Accreditation New Zealand \\
ILAC & International Laboratory Accreditation Cooperation \\
JCRB & Joint Committee of Regional Metrology Bodies \\
KCDB & Key comparison database (of the BIPM)\\
MOU & Memorandum of Understanding \\
MQC & MSL Quality Council \\
MRA & Mutual Recognition Arrangement \\
MSL & Measurement Standards Laboratory of New Zealand \\
NMIA & National Measurement Institute, Australia \\
QMS & Quality Management System \\
RMO & Regional Metrology Organisation \\
SI & Système International d'Unités (International System of Units) \\
TCM & Technical Competency Matrix \\
\hline 
\end{tabular} 
}
\end{center}
clearpage

% ----- Label for the very last page, so that we can do "page x of N"
\section{References}

% The next two lines prevent 'thebibliography' from generating the title 'References' again
\begingroup
\renewcommand{\section}[2]{}%

\begin{thebibliography}{9}
\bibitem{MRA_1999} Comit\'e International des Poids et Mesures, Mutual recognition of national measurement standards and of calibration and measurement certificates issued by national metrology institutes, Paris, 14 October 1999 (\url{http://www.bipm.org/en/cipm-mra/}).
\bibitem{Metre_convention} Metre Convention (Convention du M\`etre), also known as the Treaty of the Metre, is an international treaty that was signed in Paris on 20 May 1875. The treaty set up an institute for the purpose of coordinating international metrology and for coordinating the development of the metric system. In 1960, the system of units it had established was overhauled and relaunched as the "International System of Units" (SI).
\bibitem{ISO_17025} New Zealand Standard, ISO-IEC 17025:2018, General requirements for the competence of testing and calibration laboratories, Standards New Zealand, Wellington, 2018.
\bibitem{MS_ACT_1992} \href{http://www.legislation.govt.nz/act/public/1992/0052/latest/whole.html}{Measurement Standards Act 1992}
\bibitem{NS_Regulations} \href{http://www.legislation.govt.nz/regulation/public/2019/0091/latest/whole.html}{National Standards Regulations}
\bibitem{MSL_Reporting_Guidelines}  \href{https://edi.callaghaninnovation.govt.nz/ws/msl/QMS/QM/Guidelines%20on%20Measurement%20Uncertainty.docx?Web=1}{Guidelines on Reporting and Publishing (in the EDI library)}
\end{thebibliography}

\endgroup	% Use input because \include creates a blank last page 
\label{LastPage}~

\end{document}
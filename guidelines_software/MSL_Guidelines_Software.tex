% --------------------------------------------------------------------
% Preamble
% --------------------------------------------------------------------
\documentclass[a4paper]{article}	

\usepackage[scaled]{helvet}
\renewcommand\familydefault{\sfdefault} 
\usepackage[T1]{fontenc}

% 'microtype' can only be used with pdflatex
\usepackage[protrusion=true,expansion=true]{microtype}
	
\usepackage[utf8]{inputenc}
\usepackage[english]{babel}
\usepackage{amsmath,amsfonts,amsthm,amssymb,bm}
\usepackage{graphicx}
\usepackage{xcolor}
\usepackage{enumitem} 	% Allows different styles in enumerate
\usepackage{ulem}		% For strikethrough text
\usepackage{rotating}	% For landscape figures

% Spacing between lines 
\usepackage{setspace}
\onehalfspacing

% ----- page layout 
\usepackage[
	a4paper,
	top=1in, bottom=1.25in, left=1.25in, right=1.25in
]{geometry}	

\setlength{\oddsidemargin}{5mm}			% Remove 'twosided' indentation
\setlength{\evensidemargin}{5mm}

% ----- To allow cross-referencing between documents
%       a common root folder must be established
%       this command encapsulates the path
\newcommand{\gsdPath}{../guidelines_software}%

% ------ To cross reference other documents
\usepackage{xr-hyper}
\usepackage[
	colorlinks=true,
	urlcolor=magenta,
	linkcolor=blue, 	% [red]
%	anchorcolor 	 % [black]
	citecolor=blue, 		% [green]
	filecolor=cyan, 	% [cyan]
	menucolor=black		% [red]
]{hyperref}

% Header and footer control
\usepackage{fancyhdr}
\setlength{\headheight}{15.2pt}
\pagestyle{fancy}

% ----- Table of contents 
\usepackage{titletoc}
\usepackage{tocloft}

\renewcommand\cftsecfont{\normalfont}
\renewcommand\cftsecpagefont{\normalfont}
\renewcommand{\cftsecleader}{\cftdotfill{\cftsecdotsep}}
\renewcommand\cftsecdotsep{\cftdot}
\renewcommand\cftsubsecdotsep{\cftdot}


% ----- Tables
\usepackage{booktabs}
\usepackage{colortbl}
\usepackage{longtable}
\usepackage{makecell}

% ----- Units formatting 
\usepackage{siunitx}

% ----- For software snipets (set up for Python) 
\usepackage{listings}
\lstset{ %
language=Python,                	% the language of the code
basicstyle=\footnotesize\ttfamily,  % the size of the fonts that are used for the code
basewidth=0.51em,
columns=fullflexible,
%stringstyle=\color{white}\ttfamily,
%numbers=left,                   	% where to put the line-numbers
%numberstyle=\tiny,      			% the size of the fonts that are used for the line-numbers
%stepnumber=1,                   	% the step between two line-numbers. If it's 1, each line
                                	% will be numbered
%numbersep=5pt,                  	% how far the line-numbers are from the code
backgroundcolor=\color{yellow!15},  % choose the background color. You must add \usepackage{color}
showspaces=false,               	% show spaces adding particular underscores
showstringspaces=false,         	% underline spaces within strings
showtabs=false,                 	% show tabs within strings adding particular underscores
%frame=single,                   	% adds a frame around the code
tabsize=4,                      	% sets default tabsize to 2 spaces
captionpos=b,                   	% sets the caption-position to bottom
breaklines=true,                	% sets automatic line breaking
breakatwhitespace=false,        	% sets if automatic breaks should only happen at whitespace
title=\lstname,                 	% show the filename of files included with \lstinputlisting;
                                	% also try caption instead of title
%escapeinside={\%*}{*)},         	% if you want to add a comment within your code
escapeinside={(*}{*)},
%morekeywords={*,...}            	% if you want to add more keywords to the set
belowskip=-1.75\baselineskip		% 1.5 seems to be best
}

% ------ Macros for this document
\definecolor{yellowMSL}{RGB}{244,182,8}

% Mark-up for small changes (a few words only)
\newcommand{\proposed}[1]{{\color{cyan}{{#1}}}}
\newcommand{\deprecated}[1]{{\color{yellowMSL}{#1}}}

% Mark-up  for paragraphs. 
\usepackage{framed}
\newcommand{\proposedbox}[1]{
\colorlet{shadecolor}{gray!10}
{\color{cyan}\begin{shaded}
	{#1}
\end{shaded}}
}%

\newcommand{\deprecatedbox}[1]{
\colorlet{shadecolor}{gray!10}
{\color{yellowMSL}\begin{shaded}
	{#1}
\end{shaded}}
}%

% --------------------------------------------------------------------
% Front page definitions (do not change this)
% --------------------------------------------------------------------
\newcommand{\HRule}[1]{\rule{\linewidth}{#1}} 	% Horizontal rule

\makeatletter							% Title
\def\printtitle{%						
    {\centering \@title\par}}
\makeatother									

\makeatletter							% Author
\def\printauthor{%					
    {\raggedright \Large \@author}}				
\makeatother							


% --------------------------------------------------------------------
% Details of what to put on the front page (Change this)
% --------------------------------------------------------------------
\title{	\Large 
	\textsc{ Measurement Standards Laboratory of New Zealand } \\ [2.0cm]			
	\HRule{2pt} \\ [0.25cm]						
	\LARGE \textbf{\uppercase{Software Development Guidelines}}	\\% Title
    \Large \textit{A supplement to the MSL Quality Manual}
	\HRule{2pt} \\ [0.25cm]		
	\Large \today			
}

\author{
{\large 
	Text in this \proposed{colour} is awaiting approval by the 
	Quality Council.\\ 
	Text in this \deprecated{colour} is deprecated 
	and awaiting approval to be deleted.
} \\[\baselineskip]
 		
Measurement Quality Council\\	
Measurement Standards Laboratory of New Zealand\\	
}

\begin{document}
\hypersetup{pageanchor=false}	% prevents a warning message
\fancyhf{}	% Header and footer are empty

% ------------------------------------------------------------------------------
% Title page
% ------------------------------------------------------------------------------
\thispagestyle{empty}		% Remove page numbering on this page
\pagestyle{plain}			% puts pages numbers back in front matter

\printtitle					% Print the title data as defined above
  	\vfill
\printauthor				% Print the author data as defined above
\newpage

\pagenumbering{roman}

\tableofcontents
\newpage

% ----- Main document now begins
\pagestyle{fancy}			% Turn on footer style control
\pagenumbering{arabic}
\hypersetup{pageanchor=true}
\setcounter{page}{1}		% Set page numbering to begin on this page

\fancyfoot[L]{Measurement uncertainty}
\fancyfoot[C]{\today}
\fancyfoot[R]{Page \thepage\ of \pageref*{LastPage}}

\setlength{\parindent}{0cm}	% No indent when starting a new paragraph
\setlength{\parskip}{\baselineskip}	% Leave a blank line between paragraphs

% ----- These files contain the document content
\section{Introduction}
Every section in MSL relies on a suite of software tools for acquiring data, processing data and presenting data. No two sections use the same set of tools; most sections have developed bespoke solutions that suit their needs; there are a few cases of software being developed externally under contract.  

The quality system imposes a common set of requirements on the software used to carry out test and calibration work: it must be fit-for-purpose, it must demonstrably meet the design requirements of the task to be performed, and the integrity of software must be maintained. 

There is also a requirement to maintain the integrity of all measurement data. 

A few relatively simple practices can be followed to improve software quality and help meet the sorts of requirement faced by MSL. There are also useful software tools available. 

When writing new software in a modern programming language, these tools and practices can be incorporated without much effort. However, much of our software is in the form of `legacy' programs: software that may be modified from time to time, but which is unlikely to be redeveloped from scratch. 

Another difficulty is that some sections have invested heavily in types of software that are designed for programming by the end-user. This inherent flexibility makes them hard to maintain: spreadsheets being the best (or worst) example. Spreadsheets are convenient for quick work, but hard to validate and notorious for hiding errors.

Given the wide variety of software in use at MSL, the most useful guidance is likely to be a collection of simple ideas that can be adapted to a wide range of applications. The purpose of this document is to provide such general guidance.


\clearpage
\section{Style and Structure}
\begin{flushright}
\textit{“If you don’t know where you’re going, any road will take you there.”} \\
-- George Harrison 
\end{flushright}


Software should be developed with end-users and future owners in mind – it should also be designed to be tested and maintained easily. This requires consideration of how software is written.

\subsection{Style}
When writing, consideration should always be given to stylistic conventions: be consistent in your use of a simple, clear, style. A worthy goal is to write software that is \textit{self-explanatory}. This can be achieved by appropriate use of: names, data structures, programming structures and idioms, layout and comments. 

\paragraph{Names}
Careful naming is important. Names for variables, functions and classes, etc, should be
\begin{itemize}
\item easily remembered
\item concise
\item descriptive (for readers)
\item consistent (with names of similar entities)
\item easy to chose 
\end{itemize} 

Languages such as Visual BASIC, Python, C++, etc, and some end-user tools (e.g., MathCAD) allow considerable freedom to choose names. 

\paragraph{Data structures} All programming languages support a few basic data structures (e.g., arrays), some allow \textit{ad hoc} structures to be defined for particular uses (e.g., object-oriented structures). Good design of data structures is important, because it can make instructions that manipulate data easier to read, understand and maintain.

\paragraph{Programming structures and idioms} Authors in languages like English, etc, use syntactical rules and common idioms to write statements that can be easily understood. The same applies to programming. Each language has its own syntax, features and quirks, but there will always be good, widely-recognised, ways of describing common tasks, which can be regarded as `best-practice'.

All programming languages have a few basic control structures (e.g., \texttt{FOR-NEXT} loops, \texttt{IF-THEN} statements, etc). While the logic of these structures is simple and universal, the way they are deployed is usually language-dependent. There are subtleties in how the language works that make certain idioms preferable. Using recognised language idioms both improves code quality and makes it easier to read and maintain. 

  

\paragraph{Layout} The way that code appears on the page (screen) is important. Often source code will work fine no matter how it is laid out, but consistent layout makes a huge difference to readability. Once again, languages tend to have different conventions. Python even incorporates layout in the syntax of its block statements, in a deliberate effort to make code easier to read.

\paragraph{Comments} Software comments are something of a necessary evil: they should be used sparingly. It is better to revise code, considering appropriate naming, layout and idioms, because doing so may make a comment unnecessary.

While changes are made to code over time, comments are often overlooked. So, comments can drift with respect to the corresponding code base. Later, it may become unclear just how accurate a comment is. Writing code that does not need many comments is a better strategy. Only details that cannot be inferred from a clear coding style should be added as (preferably) short local comments.

\clearpage

\appendix	% Changes the style of the headings for sections
% \section{Abbreviations}
\begin{center}
{\renewcommand*{\arraystretch}{1.4}
\begin{tabular}{p{14.07em}p{25em}}
	\rowcolor[rgb]{ 0,  0,  0} 
	\textcolor[rgb]{ 1,  1,  1}{\textbf{Abbreviation}} & 
	\textcolor[rgb]{ 1,  1,  1}{\textbf{Stands For}} \\
APMP & Asia Pacific Metrology Programme \\ 
BIPM & Bureau International des Poids et Mesures \\ 
CC & Consultative Committee of the BIPM \\ 
CGPM & Conf\'erence G\'en\'erale des Poids et Mesures \\ 
CIPM & Comit\'e International des Poids et Mesures \\
CMC & Calibration and Measurement Capability \\
EDRMS & Electronic Document and Records Management System \\
EU & European Union \\
IANZ & International Accreditation New Zealand \\
JCRB & Joint Committee of Regional Metrology Bodies \\
MOU & Memorandum of Understanding \\
MQC & MSL Quality Council \\
MRA & Mutual Recognition Agreement \\
MSL & Measurement Standards Laboratory of New Zealand \\
NMIA & National Measurement Institute, Australia \\
QMS & Quality Management System \\
RMO & Regional Metrology Organisation \\
SI & Système International d'Unités (International System of Units) \\
TCM & Technical Competency Matrix \\
\hline 
\end{tabular} 
}
\end{center}
clearpage

% ----- Label for the very last page, so that we can do "page x of N"
\section{References}

% The next two lines prevent 'thebibliography' from generating the title 'References' again
\begingroup
\renewcommand{\section}[2]{}%

\begin{thebibliography}{99}
\bibitem{MSL_Quality_Manual} MSL Quality Manual (in the EDI \href{https://edi.callaghaninnovation.govt.nz/ws/msl/QMS/QM?Web=1}{Quality Manual} library)

\end{thebibliography}
\endgroup	% Use input because \include creates a blank last page 
\label{LastPage}~

\end{document}
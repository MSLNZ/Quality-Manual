\section{Style and Structure}
\begin{flushright}
\textit{“If you don’t know where you’re going, any road will take you there.”} \\
-- George Harrison 
\end{flushright}


Software should be developed with end-users and future owners in mind – it should also be designed to be tested and modified easily. All of which requires consideration about how software will be structured and written.

\subsection{Style and idiom}
Writing software, writing prose, or even writing mathematical formulae, has much in common.

Any language (English, French, mathematics, etc) has syntactical rules and common idioms. We work within these to create statements that others can readily understand. A programming language is much the same. Although each language has its features and quirks, there will be good ways of describing operations and there will be awful alternatives; there will be ‘best-practice’ and there will be ghastly ‘hacks’. When writing software, consideration should always be given to another reader (or even the author at some distant point in time): be consistent, express yourself using a simple, clear style.

A worthy goal is to write software that is self-explanatory. 

\paragraph{Names}
Procedural languages (such as Visual BASIC, Python, C++, etc) and some end-user tools (e.g., MathCAD) allow considerable freedom when naming things. It is important to exploit this freedom and make software that is readable. Names should be concise and descriptive: a reader should easily keep track of the meanings of different variables, functions and classes, etc.

\paragraph{Layout} ???

\paragraph{Comments}
Somewhat counter-intuitively, comments in software are something of a necessary evil: they are needed, but should be used as sparingly as possible. It is always better to re-phrase the code itself, using appropriate naming, layout and idioms, if doing so will make a comment unnecessary.
Often there will be stylistic conventions already defined for a language. These should be used, because readers are more likely to be familiar with them. If there are alternatives styles, then choose one and stick to it. Be consistent: don’t go back and forth between one style and another.  


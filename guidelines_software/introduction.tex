\section{Introduction}
\begin{flushright}
\textit{“If you don’t know where you’re going, any road will take you there.”} \\
-- George Harrison 
\end{flushright}

MSL relies heavily on software to control experiments and to acquire, process and present data. Most technical sections develop bespoke solutions that suit their needs.\footnote{It is becoming increasingly common to use `data' to refer to all forms of digital objects, so software and publication documents are also `data'.} 

Software must meet objective requirements set to accomplish a specific task. This will be verified during validation of a technical procedure. The quality system also requires the integrity of validated software and measurement data to be controlled. 

These are rather modest requirements for assets that represent a very significant investment of MSL's time to develop and maintain.  Perhaps unsurprisingly, problems arising from insufficient maintenance of software are rife. 

A lot of MSL software consists of `legacy' programs that are only occasionally used. Specification of requirements, documentation and software testing are often lacking, making maintenance difficult. Some sections have also invested heavily in types of software that are inherently difficult to maintain at the level required for calibration work (spreadsheets being a good example: convenient for quick work, but notorious for hiding errors and allowing inadvertent changes to creep in). 

Another problem is the platform required to run legacy software. IT systems evolve much faster than MSL would wish, so older hardware, and older software tools (e.g., compliers) need to be operated after they would have retired elsewhere. In many cases, these older systems are retained because not enough is known to allow migration.

There are general practices that can improve the reliability and maintainability of software. This guide recommends a planned and structured approach to software development. The objective is not merely to produce an executable application, it is also to provide: a clear statement of  requirements, an explanation of the design adopted to meet those requirements and a suite of tests that verifies the software's performance against requirements. It is essential that any part of a software project can be examined and understood by someone other than the author.  

We recommend that software development be undertaken in active consultation among members of the technical section, or wider MSL. The quality of one individual's work will benefit from a review by someone else. Professional developers use regular reviews to both detect errors and improve the quality of their work. These reviews occur regularly, not just at the end of a project.   



\section{Introduction}
Every section in MSL relies on a suite of software tools for acquiring data, processing data and presenting data. No two sections use the same set of tools; most sections have developed bespoke solutions that suit their needs; there are a few cases of software being developed externally under contract.  

The quality system imposes a common set of requirements on the software used to carry out test and calibration work: it must be fit-for-purpose, it must demonstrably meet the design requirements of the task to be performed, and the integrity of software must be maintained. 

There is also a requirement to maintain the integrity of all measurement data. 

A few relatively simple practices can be followed to improve software quality and help meet the sorts of requirement faced by MSL. There are also useful software tools available. 

When writing new software in a modern programming language, these tools and practices can be incorporated without much effort. However, much of our software is in the form of `legacy' programs: software that may be modified from time to time, but which is unlikely to be redeveloped from scratch. 

Another difficulty is that some sections have invested heavily in types of software that are designed for programming by the end-user. This inherent flexibility makes them hard to maintain: spreadsheets being the best (or worst) example. Spreadsheets are convenient for quick work, but hard to validate and notorious for hiding errors.

Given the wide variety of software in use at MSL, the most useful guidance is likely to be a collection of simple ideas that can be adapted to a wide range of applications. The purpose of this document is to provide such general guidance.



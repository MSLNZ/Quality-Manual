\section{Introduction}
MSL relies heavily on software to control experiments and for acquiring, processing and presenting data. Most technical sections develop bespoke solutions that suit their needs. 

The quality system imposes requirements on software used to carry out test and calibration work, but does not set expectations about software quality. To satisfy the quality system, software must demonstrably meet the requirements of the task, which will be verified during validation of a technical procedure. The quality system also requires the integrity of both software and measurement data to be strictly controlled. 

These are rather modest requirements for assets that represent a very significant investment of MSL's time to develop and maintain.  Perhaps unsurprisingly, problems arising from insufficient maintenance of software are rife within the Laboratory. 

Much MSL software consists of `legacy' programs that are used occasionally. Specification of requirements, documentation and software testing are often lacking, making maintenance difficult. Some sections have invested heavily in types of software that are inherently difficult to maintain at the level required for calibration work: spreadsheets being a good example. Spreadsheets are convenient for quick work, but notorious for hiding errors and renowned for allowing inadvertent changes to creep in. 

Another problem is the platform required to run legacy software. IT systems evolve much faster than MSL would wish, so older hardware, and older software tools (e.g., compliers) need to be operated after they would have retired elsewhere. In many cases, these older systems are retained because not enough is known to allow migration.

This document has been prepared to present techniques that can improve the reliability and maintainability of software. While recognising that MSL has substantially different requirements to most commercial and even scientific organisations, and that a variety of software development systems are in use, there are general practices that can improve software quality. There are also useful tools available. 

This guide recommends a structured approach to software development. The objective is not merely to produce an executable application, it is also to provide: a clear statement of the requirements, an explanation of the design to meet those requirements and a suite of tests that verifies the software's performance against requirements. It is desirable that any part of a software project can be examined and understood by someone other than the author. This facilitates re-use of software and maintenance. 

We recommend that software development be undertaken in active consultation among members of the technical section, or wider MSL. As in most quality system activities, the quality of one individual's work will benefit from a review by someone else. Professional developers use regular reviews to both detect errors and improve the quality of their work. These reviews occur throughout the development cycle, not just at the end. MSL should do this too.  



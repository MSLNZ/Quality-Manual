\section{Proficiency testing}
This guide is concerned with small proficiency tests (PTs) organised by MSL. The ISO/IEC~17043:2010 standard covers general requirements for proficiency testing and is similar in structure to ISO/IEC 17025:2017. However, ISO/IEC~17043 is intended for organisations that routinely organise proficiency testing, whereas MSL offers what are called `small inter-laboratory comparisons':\footnote{In this document we take \textit{`proficiency test'} and \textit{`inter-laboratory comparison'} to be synonymous.} proficiency tests that are not offered by an accredited provider and typically have between two to seven participants. 

Advice about running small inter-laboratory comparisons, for organisations accredited to ISO/IEC 17025 but not ISO/IEC~17043, is given in a European Accreditation guidelines document \cite{EA-4/21}. This is intended to help laboratories identify those aspects of ISO/IEC~17043 that are important for small PTs and which are not already covered by ISO/IEC 17025. The comments in EA-4/21 and reference \cite{Milde:2020} may also be helpful to accreditation bodies when assessing the results of small PTs as part of an accreditation process.

\section{Small proficiency tests}
The requirements of ISO/IEC~17043 considered relevant to the organisation of a small proficiency tests are summarised in Section~6 of the EA Guidelines \cite{EA-4/21}.

A 17025-compliant quality management system (QMS) can help to support small PTs.

The control of documents, review of requests, tenders and contracts, sub-contacting, and purchasing may all be covered by the QMS. Similarly, impartiality, confidentiality, complaints, non-conforming work, improvements and corrective actions will all be covered by an existing 17025-compliant QMS. Control of records and data will be subject to the usual quality controls.

Internal audits should be extended to cover PTs. The Management Review should also look at the organisation of PTs and consider the efficiency of operating small PTs. 

The competencies of staff should be assessed. Method-related competence would usually be recognised for routine laboratory work.

Equipment and environmental considerations would usually be covered by requirements already part of maintaining the 17025 scope of accreditation.

\paragraph{PT design: }
A PT must be thoroughly planned (see \S\ref{ss:PT_design}). Participants must be well-informed about the purpose of the PT, to allow them to decide whether it is relevant to their operations.

\paragraph{Selection and preparation of test items:}
Test items should be evaluated before the PT. Criteria should be established and the process of characterisation, stability testing, etc, should be documented. 

\paragraph{Instructions:}
Participants should be informed about the PT and its timetable well before it commences. 

\begin{itemize}
	\item Requirements for handling test items
	\item Information about test items and test methods
	\item PT time schedule
	\item Information about preparing items for testing
	\item Other instructions and informations, as needed, such as safety considerations
	\item Instructions on the recording and reporting of measurement results (i.e., formatting, uncertainty, etc)
	\item Deadlines for the submission of results
	\item Contact details for the PT organiser
\end{itemize}

\paragraph{Packaging, distribution, and handling of items:}
The requirements for distributing PT items safely must be considered. Suitable packaging and item labelling is required. Consideration should be given to safety and compliance with legal regulations. 

The PT protocol should ensure that records are kept of item distribution: 
\begin{itemize}
	\item Date of arrival
	\item Quality assurance checked by participant
	\item Date(s) of measurement
	\item Date of dispatch to organiser
\end{itemize}
 
\paragraph{Data processing:}
The organiser will be responsible for carrying out an appropriate analysis of all PT data. See references \cite{Milde:2020}, \cite{BSI:repeatability} and \cite{BSI:proficiency}.

\paragraph{Determination of PT reference value:}
This will not usually be an issue for MSL, because our measurement capability would usually be significantly better than participants. However, see \cite{Milde:2020}.
 
\paragraph{Evaluation of performance:}
Something like a z-score or En number.

\paragraph{Final reporting:}
Reporting must include data covering all participants and their performances. 

Reporting should be done promptly after completion of the PT (keeping to a previously advertised timetable).

The report should cover:

\begin{itemize}
	\item A unique identifier of the report (e.g., a number and name) 
	\item Contact details of the organiser
	\item A statement about the confidentiality of information reported
	\item A description of the PT, with dates and a description of the test items used
	\item Participant details
	\item Participant results
	\item Analysis of results
	\item Advise on the interpretation of reported statistics 
	\item Comments by the organiser arising from the results
	\item Information about appealing the results 
\end{itemize} 

\section{Relations between ISO/IEC~17043 and the MSL Quality System}
This section considers the ISO/IEC~17043 standard and, where appropriate, associates standards clauses with sections in the MSL Quality Manual. 

Sections 1 through 3 of ISO/IEC~17043 deal with: Scope, Normative references and Terms and definitions; they are not discussed below. Section~4 concerns Technical requirements and Section 5 Management requirements: these sections are considered.

\subsection{Technical requirements}

Technical requirements of ISO/IEC~17043 are partially met by MSL's quality control over calibration and measurement capabilities. There might be new competencies to be managed in our competency matrix. Such as, people competent to design PT protocols or validate designs, people to work through the data processing and people to check the results. These are similar distinctions to our existing A+R+W+C classifications. For the purposes of small PTs, we may fall back on those competencies but, if PTs become more routine, the competencies involved should be regarded as different and evidence collected to support individual claims.

There should be a data management plan for safeguarding and managing digital records (would be similar to our own calibration work).

In the following tables, the first column, labelled `Clause', refers to a clause in ISO/IEC~17043 and the second column, labelled `cross-reference(s)' refers to the MSL Quality Manual.

\begin{center}
{\renewcommand*{\arraystretch}{1.4}
\begin{tabular}{p{1em}p{10em}p{12em}p{16em}}
	\rowcolor[rgb]{ 0,  0,  0} 
	\textcolor[rgb]{ 1,  1,  1}{} & 
	\textcolor[rgb]{ 1,  1,  1}{\textbf{Clause}} & 
	\textcolor[rgb]{ 1,  1,  1}{\textbf{Cross references(s)}} &
	\textcolor[rgb]{ 1,  1,  1}{\textbf{Comments}} \\
4.1 & General & None &   \\
4.2	& Personnel &
Competencies, Training and Development \cite[\S\ref*{QM-s:competencies_professional_development}]{MSL_Quality_Manual} & 
New competency categories may be needed: PT planning; data processing and evaluate results; authorise final report \\
4.3	& Equipment, etc & Handled by technical sections	& Already covered \\
4.4	& Design of PTs	& None	& There should be a PT plan/protocol that meets the requirements of this clause. (See \S\ref{ss:PT_design}) \\
4.5	& Choice of method	& None	& I don't think this will apply to physical metrology PTs \\
4.6	& Operation of PTs	& Inwards-Outwards goods \cite[\S\ref*{QM-s:inwards_outwards_goods}]{MSL_Quality_Manual} & 
Some of this will also be in the protocol document. There are also matters of policy regarding shipping items. \\
4.7	& Data analysis	& None	& We will need to validate data that is sent to us; a protocol document should already determine how data will be analysed.\\
4.8	& Reports	& None	& PT reports will be a new reporting category that will need to be managed in a similar way to our calibration reports. There are reporting requirements in this clause.\\
4.9	& Communication	& None	& We will need a policy about communications with participants.\\
4.10	& Confidentiality	& Impartiality \cite[\S\ref*{QM-sssp:impartiality}]{MSL_Quality_Manual} and Confidentiality \cite[\S\ref*{QM-sss:confidentiality}]{MSL_Quality_Manual} & Privacy policy should be considered  (e.g., would we need to share information with an accreditation body?)

\end{tabular} } 
\end{center}

\newpage
\subsection{Management requirements}
The management requirements are almost identical to 17025. We will need to identify the types of activity, records to be managed, etc, but really the existing management system would just be expanded a little to cover these new things.

\begin{center}
{\renewcommand*{\arraystretch}{1.4}
\begin{longtable}{p{1em}p{10em}p{12em}p{16em}}
	\rowcolor[rgb]{ 0,  0,  0} 
	\textcolor[rgb]{ 1,  1,  1}{} & 
	\textcolor[rgb]{ 1,  1,  1}{\textbf{Clause}} & 
	\textcolor[rgb]{ 1,  1,  1}{\textbf{Cross references(s)}} &
	\textcolor[rgb]{ 1,  1,  1}{\textbf{Comments}} \\
5.1	& Organisation	& Quality Manual  \cite[\S\ref*{QM-s:organisation}]{MSL_Quality_Manual}  \\
5.2	&Management system	& This manual	&Would need to explicitly mention our commitment to quality in PTs \\
5.3	& Document control	& Document control\cite[\S\ref*{QM-s:documents_and_document_control}]{MSL_Quality_Manual} & We will need to issue protocols/plans in a similar way to TPs. Also, what is filed centrally and what is maintained within sections.\\
5.4	& Review of requests	& Review of requests and contracts \cite[\S\ref*{QM-s:requests_and_tenders}]{MSL_Quality_Manual} . 
& We need to keep a record of MSL reviews of a request to carry out a PT or of an internal MSL initiative about a PT.\\
5.5	& Subcontracting	& None	& Not needed? \\
5.6	& Purchasing	& None	& Not needed? \\
5.7	& Customer service	& Assessing client satisfaction \cite[\S\ref*{QM-ss:client_satisfaction}]{MSL_Quality_Manual} & Should be OK if we solicit feedback from PT participants in the same way as we survey client satisfaction for calibration work. \\
5.8	& Complaints	& Complaints \cite[\S\ref*{QM-ss:complaints}]{MSL_Quality_Manual} & OK\\

5.9	& Nonconforming work	& See \cite[\S\ref*{QM-ss:metrological_reports}]{MSL_Quality_Manual} and \cite[\S\ref*{QM-ss:improvement_requests}]{MSL_Quality_Manual} &
This is essentially our re-issue of reports and our IRF process to address problems \\
5.10	& Improvement	& Improvements  \cite[\S\ref*{QM-s:improvements}]{MSL_Quality_Manual}  \\
5.11	&Corrective actions	& Improvement requests  \cite[\S\ref*{QM-ss:improvement_requests}]{MSL_Quality_Manual} & The R\&O register is also available \\
5.12	& Preventative actions	& Improvement requests  \cite[\S\ref*{QM-ss:improvement_requests}]{MSL_Quality_Manual} & The R\&O register is also available \\
5.13	& Records control	& Documents and document control \cite[\S\ref*{QM-s:documents_and_document_control}]{MSL_Quality_Manual} & OK\\
5.14	& Internal audit	& Internal audits \cite[\S\ref*{QM-ss:internal_audit}]{MSL_Quality_Manual} & OK \\
5.15	& Management review	& Management review \cite[\S\ref*{QM-ss:management_review}]{MSL_Quality_Manual} & OK \\	
\end{longtable} } 
\end{center}

\subsection{Confidentiality}
The same level of confidentiality that applies to calibration work shall be used for proficiency testing. 

A different confidentiality statement will be needed. 

The following is a possibility
\begin{quote}
\it
All information supplied by you as part of a proficiency testing program is treated as confidential, except where we are required by law or governmental authority, or where authorised by you. In reports and publications produced by MSL that will be seen by other participants, your data will be identified by numbers or symbols that maintain confidentiality.
\end{quote}

Exceptions are possible and should be flagged to the participant (at the outset):
\begin{itemize}
\item	An interested party may require PT results to be provided directly to them (e.g., IANZ). This should be indicated to participants at the time of registration, or evidence that it is acceptable should be collected.
\item	A regulatory authority may require PT results to be provided directly to them. This should be indicated to participants at the time of registration, or evidence that it is acceptable should be collected.

\end{itemize}

%\remark{We will need to make sure that each participant sees this}

\paragraph{Participant identifiers:}\mbox{}\\
There may be a need to label results with anonymous IDs when reporting results. We could create a pool of IDs before registration begins and attribute them randomly and when registration closes, keeping a list of participants vs IDs for internal use.
 

\subsection{Proficiency test design}
\label{ss:PT_design}
A planning document must be prepared before every PT. The plan is a requirement of 17043 (clause 4.4.1.3).  The document must address the objectives and PT scheme design. 


There are 21 points that must be covered in the plan. The headings are as follows, with some brief comments (\S6.2.3 of the EA Guidelines summarises the minimum requirements for small PTs \cite{EA-4/21}) 

\begin{enumerate}[a)]
\item	\textbf{PT provider} \\
Measurement Standards Laboratory of New Zealand (MSL)

\item	\textbf{PT coordinator} \\
(Name, address, contact details). For example,

Peter Saunders\\
Measurement Standards Laboratory\\
Tel.: 04 931 3143\\
email: peter.saunders@measurement.govt.nz

\item	\textbf{Subcontracted activities} \\
There are probably none, but this could relate to sample preparation and expert statistical services.

\item	\textbf{Criteria to be met for participants} \\

\textit{Participation is restricted to New Zealand laboratories only.  Please contact your country's National Metrology Institute for information about proficiency tests available in your country.}

\vspace{\baselineskip}
Note, when registering, we might want participants to supply information about their circumstances  e.g., to return answers to a series of questions.\footnote{Whether to restrict participation to 17025-accredited laboratories may also be considered. Why would we allow non-accredited labs to participate (this is risky), perhaps only those intending to become accredited should be allowed (but how would we know)?}

\item	\textbf{Expected number of participants} \\
To be sure the PT can be properly resourced. 

\item	\textbf{Information about the measurand} \\ \label{l:measurands}%
This should cover what participants are expected to measure or test and may also include the aim of the PT. The PT protocol will also include this information.  This information should be reviewed and validated internally.

\item	\textbf{The expected range of measured values} \\ \label{l:ranges}%
It is important to consider the diversity of responses that may be submitted. It pays to think about how labs might behave (unintended behaviour). This will determine acceptable data processing strategies (which should be decided before the results are reviewed).

\item	\textbf{The major sources of measurement error} \\
This builds on \ref{l:measurands} and \ref{l:ranges}. It is asking about the main features of the measurement model, which will probably be influence factors for all participants.


\item	\textbf{Requirements for quality control, storage and distribution of items} \\  
 \label{l:QA_requirements} 
These are the things that MSL must take responsibility for (but, see \ref{l:stability} below). Test items must meet pre-defined quality requirements and must not become compromised during the PT. How items will be packed and shipped and how they will be checked by MSL during the PT need to be considered.


\item	\textbf{Collusion between participants or falsification of results}  \\
This should cover the procedure to be followed if collusion or falsification is suspected. We should state that such behaviour will not be tolerated and explain what will happen if we suspect that it has occurred.\footnote{The statement should allow MSL to drop a participant when there are some grounds for concern, rather than placing a burden of proof on MSL to show that someone has behaved inappropriately.} 
 
It may be possible to ask for extra information with results that would make it easier for us to detect bad behaviour (e.g, raw data immediately following a measurement).


\item	\textbf{Information for participants and the time schedule}  \\
This is usually all covered in the Technical Protocol.

\item	\textbf{For continuous PTs} \\ 
A timetable for distribution of items to participants, the deadline for returns.

\item	\textbf{Supplementary information on methods or procedures} \\
This may or many not be needed.

\item	\textbf{Procedures to establish stability of PT items} \\ \label{l:stability}%
This is the quality assurance procedures each participant should use to establish stability of the items they receive. It is distinct from quality assurance MSL may use \ref{l:QA_requirements}.

\item	\textbf{Standardised reporting formats} \\
A description of any requirements for the format when reporting results. 

\item	\textbf{Analysis of results} \\ \label{l:statistics}%
A detailed description of the analysis that will be used to analyse results.
This is an important technical question. Like \ref{l:measurands}, this should be validated internally before the PT begins.

\item	\textbf{Traceability} \\
How is traceability obtained?

\item	\textbf{Performance criteria} \\
How will the performance of participants be evaluated (e.g., En values)? This relates to the purpose of the PT and what is actually being tested. This is also related to \ref{l:statistics}. 


\item	\textbf{Interim reporting} \\
If interim reports, or other information will be given to participants before the end of the PT this should be noted. 

Think about how participants may be informed promptly after they have submitted a result. We might issue an interim report with data that is not a full calibration report (hence unsigned). If so, general quality principles should still apply: it should be prepared by one person and checked by another, with some record of checking retained. 

The publication clearance process might be used to cover the whole set of reports for on PT. 

\item	\textbf{Publication of information} \\
The extent to which participant results, and conclusions based on the outcome of the PT, are to be made public should be noted. 

Will there be a final overarching report? If so, will participants maintain their anonymity in this report (we pledge this in the confidentiality statement)? We might list participant names at the beginning of the report, but not identify the results for each participant. We might prepare a report that can only be distributed among participants.

\item	\textbf{Lost or damaged test items} \\
What actions will be taken if items are lost or damaged? Who is financially responsible? Has insurance been considered? Is it possible to mitigate any of the various risks. If a PT can be seriously impacted, how will communication with participants be handled, will there be refunds, etc?

\end{enumerate}


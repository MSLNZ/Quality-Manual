\section{Proficiency testing}
This guide concerns the ISO/IEC 17043:2010 standard on the general requirements for proficiency testing. That standard is distinct from ISO/IEC 17025 but there are many similarities. The purpose of this document is to present an overview of the clauses in ISO/IEC 17043 and, where appropriate, associate them with sections in our ISO/IEC 17025 Quality Manual. 
Sections 1 through 3 of ISO/IEC 17043 concern Scope, Normative references and Terms and definitions. Sections 4 and 5 are considered in greater detail below. Section 4 concerns Technical requirements and section 5 Management requirements.

\subsection{Technical requirements}

Technical requirements are partially met by MSL's quality control over calibration and measurement capabilities. There are new competencies, which could be managed in our TCM. We need people to design PT protocols and validate the designs, people to work through the data processing and people to check the results. These are similar distinctions as our existing A+R+W+C classifications, but the competencies involved are substantially different.

\begin{center}
{\renewcommand*{\arraystretch}{1.4}
\begin{tabular}{p{1em}p{10em}p{12em}p{16em}}
	\rowcolor[rgb]{ 0,  0,  0} 
	\textcolor[rgb]{ 1,  1,  1}{} & 
	\textcolor[rgb]{ 1,  1,  1}{\textbf{Clause}} & 
	\textcolor[rgb]{ 1,  1,  1}{\textbf{Cross references(s)}} &
	\textcolor[rgb]{ 1,  1,  1}{\textbf{Comments}} \\
4.1 & General & None &   \\
4.2	& Personnel & Section 5 & 
New TCM categories may be needed: PT planning; data processing and evaluate results; authorise final report \\
4.3	& Equipment, etc & Handled by technical sections	& Already covered \\
4.4	& Design of PTs	& None	& This part requires consideration. There should be a PT plan/protocol that meets the requirements of this clause. (See next section) \\
4.5	& Choice of method	& None	& I don't think this will apply to physical metrology PTs \\
4.6	& Operation of PTs	& Inwards-Outwards goods section 7. & 
Some of this will also be in the protocol document but there are also matters of policy regarding communications and shipping items. \\
4.7	& Data analysis	& None	& We will need to validate data that is sent to us; a protocol document should already determine how data will be analysed.\\
4.8	& Reports	& None	& PT reports will be a new reporting category that will need to be managed in a similar way to our calibration reports. There are reporting requirements in this clause.\\
4.9	& Communication	& None	& We will need a policy about communications with participants.\\
4.10	& Confidentiality	& Impartiality and Confidentiality & Quality Manual sections will apply: \cite[\S\ref*{QM-sss:confidentiality}]{MSL_Quality_Manual} and \cite[\S\ref*{QM-sssp:impartiality}]{MSL_Quality_Manual}.

\end{tabular} } \end{center}

\subsection{Management requirements}
The management requirements are almost identical to 17025. We will need to identify the types of activity, records to be managed, etc, but really the existing management system would just be expanded a little to cover these new things.

\begin{center}
{\renewcommand*{\arraystretch}{1.4}
\begin{tabular}{p{1em}p{10em}p{12em}p{16em}}
	\rowcolor[rgb]{ 0,  0,  0} 
	\textcolor[rgb]{ 1,  1,  1}{} & 
	\textcolor[rgb]{ 1,  1,  1}{\textbf{Clause}} & 
	\textcolor[rgb]{ 1,  1,  1}{\textbf{Cross references(s)}} &
	\textcolor[rgb]{ 1,  1,  1}{\textbf{Comments}} \\
5.1	& Organisation	& Quality Manual  \cite[\S\ref*{QM-s:organisation}]{MSL_Quality_Manual} & OK \\
5.2	&Management system	&This manual	&Would need to explicitly mention our commitment to quality in PTs \\
5.3	&Document control	&\cite[\S\ref*{QM-s:documents_and_document_control}]{MSL_Quality_Manual} on document control applies	& We will need to issue protocols/plans in a similar way to TPs. Also, what is filed centrally and what is maintained within sections.\\
5.4	& Review of requests	& Quality Manual \cite[\S\ref*{QM-s:requests_and_tenders}]{MSL_Quality_Manual} . 
& We need to keep a record of our review of a request to carry out a PT.\\
5.5	& Subcontracting	& None	& Not needed? \\
5.6	& Purchasing	& None	& Not needed? \\
5.7	& Customer service	& \cite[\S\ref*{QM-ss:client_satisfaction}]{MSL_Quality_Manual} to assess satisfaction	& Should be OK \\
5.8	& Complaints	& Quality Manual  \cite[\S\ref*{QM-ss:complaints}]{MSL_Quality_Manual} & OK\\

5.9	& Nonconforming work	& \cite[\S\ref*{QM-ss:metrological_reports}]{MSL_Quality_Manual} and \cite[\S\ref*{QM-ss:improvement_requests}]{MSL_Quality_Manual} &
This is essentially our re-issue of reports and our IRF process to address problems \\
5.10	& Improvement	& Quality Manual  \cite[\S\ref*{QM-s:improvements}]{MSL_Quality_Manual}  \\
5.11	&Corrective actions	& Quality Manual  \cite[\S\ref*{QM-ss:improvement_requests}]{MSL_Quality_Manual} & The R\&O register is also available \\
5.12	& Preventative actions	& Quality Manual  \cite[\S\ref*{QM-ss:improvement_requests}]{MSL_Quality_Manual} & The R\&O register is also available \\
5.13	& Records control	& Quality Manual  \cite[\S\ref*{QM-s:documents_and_document_control}]{MSL_Quality_Manual} & OK\\
5.14	& Internal audit	& Quality Manual  \cite[\S\ref*{QM-ss:internal_audit}]{MSL_Quality_Manual} & OK \\
5.15	& Management review	& Quality Manual  \cite[\S\ref*{QM-ss:management_review}]{MSL_Quality_Manual} & OK \\	
\end{tabular} } 
\end{center}

\subsection{Confidentiality}
The same level of confidentiality that applies to calibration work shall be used for proficiency testing. 

A different confidentiality statement will be needed. 

The following is a possibility
\begin{quote}
\it
All information supplied by you as part of a proficiency testing program is treated as confidential, except where we are required by law or governmental authority, or where authorised by you. In reports and publications produced by MSL, your data will be identified by numbers or symbols that maintain confidentiality.
\end{quote}

Exceptions are possible and should be flagged to the participant (at the outset):
\begin{itemize}
\item	An interested party may require PT results to be provided directly to them (e.g., IANZ). This should be indicated to participants at the time of registration, or evidence that it is acceptable should be collected.
\item	A regulatory authority may require PT results to be provided directly to them. This should be indicated to participants at the time of registration, or evidence that it is acceptable should be collected.

\end{itemize}

\remark{We will need to make sure that each participant sees this}

\paragraph{Participant identifiers:}\mbox{}\\
There will be a need to label results with anonymous IDs when reporting results. Cynthia could be given a pool of IDs before registration begins. She will attribute them and when registration closes, she will provide a list of participants vs IDs back to the team.

\subsection{Proficiency test design}

A planning document must be prepared before every PT. This document must address the objectives and PT scheme design. The plan is a requirement of 17043 (clause 4.4.1.3). 
There are 21 points that must be addressed in the plan. The following schema may help to make sure that all are covered. 

\paragraph{Plan headings:}\mbox{}\\

\begin{enumerate}[a)]
\item	PT provider \\
Measurement Standards Laboratory of New Zealand (MSL)

\item	PT coordinator \\
(Name, address, contact details)

\item	Subcontracted activities \\
(None)

\item	Criteria to be met for participants \\
(17025-accredited laboratory) \\
Criteria to be met for participants:
(accredited laboratory?) This is a question of policy. Why would we allow non-accredited labs to participate (this is risky), perhaps only those intending to become accredited (how would we know?)

Note too that at the moment of registration some teams may want the participant to supply information about their circumstances. We might need them to return answers to a series of questions.

\item	Expected number of participants \\
(...) This is just to make sure we have properly resourced it: if suddenly this number were exceeded you would know to check if we had the capacity to go ahead

\item	Information on measurands, including what the participants are to measure or test \\
(...) This will go in the protocol, but it is important that the team knows (and has had this validated internally). This is the place to state what the aim of the PT is too.

\item	The range of measured values expected \\
(...) Same as f). Note here it pays to think about how labs might behave (unintended behaviour) and document assumptions and planned responses. We should not be adapting the rules as we see the data coming in. 

\item	The major sources of measurement error expected in this type of measurement \\
(...) Same as f) – it is asking whether we have thought about the measurement model

\item	Requirements for quality control, storage and distribution of items used in the PT \\
(...) Shipping, etc, as well as QA to be sure the item has not been compromised. These are things that MSL will assume responsibility for (but see (n) below).

\item	Reasonable precautions to prevent collusion between participants or falsification of results, and procedures to be employed if collusion or falsification is suspected  \\
(...) This will depend on the PT, but we should have a statement that such behaviour will not be tolerated and what will happen if we suspect it (level of evidence?). It may be possible to ask for extra things that would make it easier for us to detect bad behaviour (e.g, raw data immediately following a measurement).

\item	The information that will be given to participants and the time schedule for the various phases of the PT \\
(...) 

\item	(For continuous PTs only) The timetable for distribution of items to participants, the deadline for returns and, where appropriate, dates on which participants will make measurements \\
(...) 

\item	Any information on methods or procedures that participants need \\
(...) 

\item	Procedures to be used to establish the stability of PT items \\
(...)  This sounds like i) but this refers to procedures that the lab will use to establish stability.

\item	Description of any standardised reporting formats to be used by participants \\
(...) 

\item	A detailed description of the statistical analysis that will be used to analyse results \\
(...)  Like f-h, this is a technical question that should be sorted and validated before the PT starts

\item	The uncertainty of any assigned values (must be traceable)
[NA] This needs rewording. The question is: how is traceability obtained by MSL?

\item	Performance criteria for participants\\
(...) E.g., En values, this is related to p). Note, we should be clear in our own minds about what the purpose of the PT is: what is the actual proficiency that is being tested. This should be documented. Knowing this will make other decisions clearer.

\item	A description of the data, interim reports or other information that will be returned to participants\\
(...) Think about how participants can be informed promptly after they have submitted a report, so that they get feedback before the final report for the PT is prepared. We might issue an interim report with data. This report is unlikely to be a full calibration report, and unsigned, but general quality principles apply so it should be prepared by one person and checked by another, with some record of checking retained. The publication clearance process might be used to cover the whole set of reports for on PT. 

\item	A description of the extent to which participant results, and conclusions based on the outcome of the PT, are to be made public\\
(...) Will there be an overall PT report? If so, participants will be anonymized (we pledge this in the confidentiality statement). However, we might name the participants at the beginning of the report (or not) and we might prepare a report that can only be distributed among participants, or not.

\item	Actions to be taken in the case of lost or damaged PT test items\\
(...) One important question is: who pays if it breaks? Has the question of insurance been considered? Also, some thought about mitigating the effects of various risks. Also, communication with clients: we will tell them if ….; refunds, etc?

\end{enumerate}

Another idea regarding confidentiality and anonymisation: the team could give Cynthia a pool of identifiers that she can distribute to people who register (the point is that Cynthia does this completely independently of the team, which mitigates of the communication mishaps leading to leaked information about the comparison). Cynthia would let the team know how she has assigned the ids when registration closes.
 

Not discussed but also important: 
\begin{enumerate}
\item	Plan for safeguarding and managing digital records.
\item	Privacy policy (we are subject to CI policy)
\end{enumerate}
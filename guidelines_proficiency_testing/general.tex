\section{Proficiency testing}
This guide is concerned proficiency testing organised by MSL. It takes the ISO/IEC 17043:2010 standard into account. That standard is distinct from ISO/IEC 17025:2017, although there are many similarities, and MSL does not currently intend to fully implement the standard to seek accreditation \cite{BSI:proficiency}. 

The 17043 standard is mainly concerned with organisations that routinely organise proficiency testing, which is not the case for MSL. The nature of the proficiency tests (PTs) that MSL offers are more in line with what has been called \textit{small interlaboratory comparisons}. Some advice about running these smaller comparisons is given in \cite{Milde:2020} and \cite{EA-4/21}

This document provides an overview of ISO/IEC 17043 and, where appropriate, associates clauses in that standard with sections in our ISO/IEC 17025 Quality Manual. 
Sections 1 through 3 of ISO/IEC 17043 deal with: Scope, Normative references and Terms and definitions; they are not discussed here. Section~4 concerns Technical requirements and section 5 Management requirements: these sections are considered below.

\subsection{Technical requirements}

Technical requirements of 17043 are partially met by MSL's quality control over calibration and measurement capabilities. There are new competencies, which could be managed in our TCM. We need people to design PT protocols and validate the designs, people to work through the data processing and people to check the results. These are similar distinctions as our existing A+R+W+C classifications, but the competencies involved are substantially different.

\begin{center}
{\renewcommand*{\arraystretch}{1.4}
\begin{tabular}{p{1em}p{10em}p{12em}p{16em}}
	\rowcolor[rgb]{ 0,  0,  0} 
	\textcolor[rgb]{ 1,  1,  1}{} & 
	\textcolor[rgb]{ 1,  1,  1}{\textbf{Clause}} & 
	\textcolor[rgb]{ 1,  1,  1}{\textbf{Cross references(s)}} &
	\textcolor[rgb]{ 1,  1,  1}{\textbf{Comments}} \\
4.1 & General & None &   \\
4.2	& Personnel & Section 5 & 
New TCM categories may be needed: PT planning; data processing and evaluate results; authorise final report \\
4.3	& Equipment, etc & Handled by technical sections	& Already covered \\
4.4	& Design of PTs	& None	& This part requires consideration. There should be a PT plan/protocol that meets the requirements of this clause. (See next section) \\
4.5	& Choice of method	& None	& I don't think this will apply to physical metrology PTs \\
4.6	& Operation of PTs	& Inwards-Outwards goods section 7. & 
Some of this will also be in the protocol document but there are also matters of policy regarding communications and shipping items. \\
4.7	& Data analysis	& None	& We will need to validate data that is sent to us; a protocol document should already determine how data will be analysed.\\
4.8	& Reports	& None	& PT reports will be a new reporting category that will need to be managed in a similar way to our calibration reports. There are reporting requirements in this clause.\\
4.9	& Communication	& None	& We will need a policy about communications with participants.\\
4.10	& Confidentiality	& Impartiality and Confidentiality & Quality Manual sections will apply: \cite[\S\ref*{QM-sss:confidentiality}]{MSL_Quality_Manual} and \cite[\S\ref*{QM-sssp:impartiality}]{MSL_Quality_Manual}.

\end{tabular} } \end{center}

\subsection{Management requirements}
The management requirements are almost identical to 17025. We will need to identify the types of activity, records to be managed, etc, but really the existing management system would just be expanded a little to cover these new things.

\begin{center}
{\renewcommand*{\arraystretch}{1.4}
\begin{tabular}{p{1em}p{10em}p{12em}p{16em}}
	\rowcolor[rgb]{ 0,  0,  0} 
	\textcolor[rgb]{ 1,  1,  1}{} & 
	\textcolor[rgb]{ 1,  1,  1}{\textbf{Clause}} & 
	\textcolor[rgb]{ 1,  1,  1}{\textbf{Cross references(s)}} &
	\textcolor[rgb]{ 1,  1,  1}{\textbf{Comments}} \\
5.1	& Organisation	& Quality Manual  \cite[\S\ref*{QM-s:organisation}]{MSL_Quality_Manual} & OK \\
5.2	&Management system	&This manual	&Would need to explicitly mention our commitment to quality in PTs \\
5.3	&Document control	&\cite[\S\ref*{QM-s:documents_and_document_control}]{MSL_Quality_Manual} on document control applies	& We will need to issue protocols/plans in a similar way to TPs. Also, what is filed centrally and what is maintained within sections.\\
5.4	& Review of requests	& Quality Manual \cite[\S\ref*{QM-s:requests_and_tenders}]{MSL_Quality_Manual} . 
& We need to keep a record of our review of a request to carry out a PT.\\
5.5	& Subcontracting	& None	& Not needed? \\
5.6	& Purchasing	& None	& Not needed? \\
5.7	& Customer service	& \cite[\S\ref*{QM-ss:client_satisfaction}]{MSL_Quality_Manual} to assess satisfaction	& Should be OK \\
5.8	& Complaints	& Quality Manual  \cite[\S\ref*{QM-ss:complaints}]{MSL_Quality_Manual} & OK\\

5.9	& Nonconforming work	& \cite[\S\ref*{QM-ss:metrological_reports}]{MSL_Quality_Manual} and \cite[\S\ref*{QM-ss:improvement_requests}]{MSL_Quality_Manual} &
This is essentially our re-issue of reports and our IRF process to address problems \\
5.10	& Improvement	& Quality Manual  \cite[\S\ref*{QM-s:improvements}]{MSL_Quality_Manual}  \\
5.11	&Corrective actions	& Quality Manual  \cite[\S\ref*{QM-ss:improvement_requests}]{MSL_Quality_Manual} & The R\&O register is also available \\
5.12	& Preventative actions	& Quality Manual  \cite[\S\ref*{QM-ss:improvement_requests}]{MSL_Quality_Manual} & The R\&O register is also available \\
5.13	& Records control	& Quality Manual  \cite[\S\ref*{QM-s:documents_and_document_control}]{MSL_Quality_Manual} & OK\\
5.14	& Internal audit	& Quality Manual  \cite[\S\ref*{QM-ss:internal_audit}]{MSL_Quality_Manual} & OK \\
5.15	& Management review	& Quality Manual  \cite[\S\ref*{QM-ss:management_review}]{MSL_Quality_Manual} & OK \\	
\end{tabular} } 
\end{center}

\subsection{Confidentiality}
The same level of confidentiality that applies to calibration work shall be used for proficiency testing. 

A different confidentiality statement will be needed. 

The following is a possibility
\begin{quote}
\it
All information supplied by you as part of a proficiency testing program is treated as confidential, except where we are required by law or governmental authority, or where authorised by you. In reports and publications produced by MSL, your data will be identified by numbers or symbols that maintain confidentiality.
\end{quote}

Exceptions are possible and should be flagged to the participant (at the outset):
\begin{itemize}
\item	An interested party may require PT results to be provided directly to them (e.g., IANZ). This should be indicated to participants at the time of registration, or evidence that it is acceptable should be collected.
\item	A regulatory authority may require PT results to be provided directly to them. This should be indicated to participants at the time of registration, or evidence that it is acceptable should be collected.

\end{itemize}

\remark{We will need to make sure that each participant sees this}

\paragraph{Participant identifiers:}\mbox{}\\
There will be a need to label results with anonymous IDs when reporting results. Cynthia could be given a pool of IDs before registration begins. She will attribute them and when registration closes, she will provide a list of participants vs IDs back to the team.

\subsection{Proficiency test design}

A planning document must be prepared before every PT. The plan is a requirement of 17043 (clause 4.4.1.3).  The document must address the objectives and PT scheme design. 


There are 21 points that must be covered in the plan. The headings are as follows, with some brief comments  

\begin{enumerate}[a)]
\item	\textbf{PT provider} \\
Measurement Standards Laboratory of New Zealand (MSL)

\item	\textbf{PT coordinator} \\
(Name, address, contact details). For example:

Peter Saunders\\
Measurement Standards Laboratory\\
Tel.: 04 931 3143\\
email: peter.saunders@measurement.govt.nz

\item	\textbf{Subcontracted activities} \\
(Probably none)

\item	\textbf{Criteria to be met for participants} \\

\textit{Participation is restricted to New Zealand laboratories only.  Please contact your country's National Metrology Institute for information about proficiency tests available in your country.}

\vspace{\baselineskip}
Note, when registering, we might want participants to supply information about their circumstances  e.g., to return answers to a series of questions.\footnote{Whether to restrict participation to 17025-accredited laboratories may also be considered. Why would we allow non-accredited labs to participate (this is risky), perhaps only those intending to become accredited should be allowed (but how would we know)?}

\item	\textbf{Expected number of participants} \\
To be sure the PT can be properly resourced. 

\item	\textbf{Information about the measurand} \\ \label{l:measurands}%
This should cover what participants are expected to measure or test and may also include the aim of the PT. The PT protocol will also include this information.  This information should be reviewed and validated internally.

\item	\textbf{The expected range of measured values} \\ \label{l:ranges}%
It is important to consider the diversity of responses that may be submitted. It pays to think about how labs might behave (unintended behaviour). This will determine acceptable data processing strategies (which should be decided before the results are reviewed).

\item	\textbf{The major sources of measurement error} \\
This builds on \ref{l:measurands} and \ref{l:ranges}. It is asking about the main features of the measurement model, which will probably be influence factors for all participants.


\item	\textbf{Requirements for quality control, storage and distribution of items} \\
These are the things that MSL must assume responsibility for (but, see \ref{l:stability} below). We must ensure that items do not become compromised. How items will be packed and shipped and how they will be checked during the PT need to be considered.


\item	\textbf{Collusion between participants or falsification of results}  \\
This should cover the procedure to be followed if collusion or falsification is suspected. We should state that such behaviour will not be tolerated and explain what will happen if we suspect that it has occurred.\footnote{The statement should allow MSL to drop a participant when there are some grounds for concern, rather than placing a burden of proof on MSL to show that someone has behaved inappropriately.} 
 
It may be possible to ask for extra information with results that would make it easier for us to detect bad behaviour (e.g, raw data immediately following a measurement).


\item	\textbf{Information for participants and the time schedule}  \\
This is usually all covered in the Technical Protocol.

\item	\textbf{For continuous PTs} \\ 
A timetable for distribution of items to participants, the deadline for returns.

\item	\textbf{Supplementary information on methods or procedures} \\
This may or many not be needed.

\item	\textbf{Procedures to establish stability of PT items} \\ \label{l:stability}%
This is about the procedures each participant will use to establish stability of the items they receive for measurement.

\item	\textbf{Standardised reporting formats} \\
A description of any requirements for the format when reporting results. 

\item	\textbf{Statistical analysis of results} \\ \label{l:statistics}%
A detailed description of the statistical analysis that will be used to analyse results.
This is an important technical question. Like \ref{l:measurands}, this should be validated internally before the PT begins.

\item	\textbf{Traceability} \\
How is traceability to the SI obtained?

\item	\textbf{Performance criteria} \\
How will the performance of participants be evaluated (e.g., En values)? This relates to the purpose of the PT and what is actually being tested. This is also related to \ref{l:statistics}. 


\item	\textbf{Interim reporting} \\
If interim reports, or other information will be given to participants before the end of the PT this should be noted. 

Think about how participants may be informed promptly after they have submitted a result. We might issue an interim report with data that is not a full calibration report (hence unsigned). If so, general quality principles should still apply: it should be prepared by one person and checked by another, with some record of checking retained. 

The publication clearance process might be used to cover the whole set of reports for on PT. 

\item	\textbf{Publication of information} \\
The extent to which participant results, and conclusions based on the outcome of the PT, are to be made public should be noted. 

Will there be a final overarching report? If so, will participants maintain their anonymity in this report (we pledge this in the confidentiality statement)? We might list participant names at the beginning of the report, but not identify the results for each participant. We might prepare a report that can only be distributed among participants.

\item	\textbf{Lost or damaged test items} \\
What actions will be taken if items are lost or damaged? Who is financially responsible? Has insurance been considered? Is it possible to mitigate any of the various risks. If a PT can be seriously impacted, how will communication with participants be handled, will there be refunds, etc?

\end{enumerate}

Another idea regarding confidentiality and anonymity: the team could give Cynthia a pool of identifiers that she can distribute to people who register (the point is that Cynthia does this completely independently of the team, which mitigates of the communication mishaps leading to leaked information about the comparison). Cynthia would let the team know how she has assigned the ids when registration closes.
 

Not discussed but also important: 
\begin{enumerate}
\item	Plan for safeguarding and managing digital records.
\item	Privacy policy (we are subject to CI policy)
\end{enumerate}
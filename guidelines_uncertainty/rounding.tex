\section{Fixed-digit indications}
\subsection{Rounding}
The rounding guidelines in NIST Good Laboratory Practice Guide \cite{GLP9}  apply when instrument readings are formatted in a fixed number of digits.

\subsubsection{Single readings}

The expression of uncertainty in a single indication should not be constrained by the format of the data.  

For example, suppose an indication is limited to integer values and that the number $101$ is observed. It is \underline{not} recommended to report the result as $101,\; U=1$. The standard uncertainty (of a uniform distribution of error between $0$ and $1$) is $0.29$. So with a coverage factor $k=2$, this result would be reported as $101.00,\; U=0.58$ (or perhaps as $101.0,\; U=0.6$).

\paragraph{Why report more digits than the instrument can display?} 
The extra digits are needed because the uncertainty of the reading, as an estimate of the quantity being measured, determines the number of significant figures required.

For the case of an instrument that displays truncated readings, a simple measurement model is \[
Y=w+E\;,
\]
where $w$ is the reading, which is known, and $E$ is the truncation error, which is unknown. The error can be described statistically, using a uniform distribution with zero mean and a range equal to the least significant digit displayed. 

The reported value is $y$, an estimate of $Y$. This is the sum of $w$ and the best estimate of $E$, which is zero.  So, \[
y=w+0\;.
\]
The uncertainty of $y$, as an estimate of $Y$, is found by combining the uncertainty associated with $w$, which is zero (we know exactly what the reading is), with the uncertainty associated with taking zero as an estimate of $E$, which is $0.29$. So,\[
u(y)= \sqrt{(0^2+0.29^2 }=0.29\;.  
\]

\subsubsection{Multiple readings}
The uncertainty associated with the mean of a series of readings with a fixed number of decimal places can be evaluated by a simple method that takes advantage of variations in the data to reduce the size of the uncertainty \cite{WILLINK_07}. Section 3 of Willink's paper describes the best way of processing data in this situation (a view).
 
This method is has been independently corroborated as best-practice by NIST.  It is available in GTC software as the function \verb|type_a.estimate_digitised()|.   The following example of use is in the GTC documentation:
\begin{quote}
\begin{verbatim}
# LSD = 0.0001, data varies between -0.0055 and -0.0057
>>> seq = (-0.0056,-0.0055,-0.0056,-0.0056,-0.0056, 
...      -0.0057,-0.0057,-0.0056,-0.0056,-0.0057,-0.0057)
>>> type_a.estimate_digitized(seq,0.0001)
ureal(-0.005627272727272727, 1.9497827808661157e-05, 10)
\end{verbatim}
\end{quote}

A slightly simplified version of the result given in \cite{WILLINK_07}, in the case where a set of differences between a reference instrument and a device under test (DUT) is averaged to determine a correction, is to assign the uncertainty in this correction as either $s/\sqrt{N}$ or $\delta/\sqrt{12}$, whichever is larger, where $s$ is the standard deviation of the $N$ differences and $\delta$ is the resolution of the DUT. This is appropriate advice for second-tier calibration laboratories provided that $N$ is at least 4.

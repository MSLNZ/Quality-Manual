\section{Inwards and outwards goods control}
\label{s:inwards_outwards_goods}
It is important that all inwards goods be tracked and delivered promptly to the appropriate staff member without loss or damage. This includes all items received for testing, calibration, examination, use on client work or otherwise borrowed or leased by MSL.

It is also important that outwards goods are properly packaged and dispatched to ensure prompt service and to avoid loss or liability due to damage.

\subsection{Inwards goods procedure}
Any member of MSL who accepts inwards goods must follow this procedure.

\subsubsection{Documentation}
\paragraph{Goods Log Book}
A ``Goods Log Book'' may be used to record the date of receipt, item, client name, recipient for each client item handled. Alternatively, this information may be recorded electronically as part of the section file system.

\paragraph{Goods Control Form}
Alternatively, a ``Goods Control Form'' may be used to record the details above.  The form may be used to record further detail, such as enumerating the contents of a package. The form will be filed in the job records in the section file system. 

A Goods Control Form must be used if goods are received damaged.

A template Goods Control Form is provided in the Quality Manual folder within the MSL Quality Management System AODocs site.

\paragraph{Goods control label}
A small, removable, self-adhesive label marked with MSL, section name and identifying the job, may be used in addition to a Goods Control Form or Goods Log Book. Where appropriate, the label may be attached to a plastic luggage tag tied to the goods or its packaging.

\subsubsection{Client goods}
Including all items received for testing, calibration, examination, use on client work or otherwise borrowed or leased by MSL. 
\begin{itemize}
\item The person who accepts the goods shall arrange transfer to an appropriate member of MSL staff.
\item The MSL staff member shall record the arrival of the goods in the Goods Log Book (or alternative as outlined above) and to inspect for damage.
\item If the goods are damaged, or need a detailed description, a Goods Control Form shall be used. If the goods may be confused with others in the workplace, or are at risk of being mislaid, they should be labelled with goods control labels. 
\item Every person having custody of goods is required to ensure safe storage and handling. 
\item If goods are accepted prior to receipt of a contract or Work Order Agreement signed by the client, then a signed contract shall be sought immediately.
\end{itemize} 

\subsection{Outwards goods procedure}
Any member of MSL with goods for dispatch must follow this procedure.

\subsubsection{Documentation}
\paragraph{Goods Log Book}
A ``Goods Log Book'' may be used to record the date of despatch, the despatcher's details and the carrier for each client item handled. Alternatively, this information may be recorded electronically as part of the section file system.

\paragraph{Goods Control Form}
Another alternative is to record details on a ``Goods Control Form'', which will be filed within job records in the section file system. 

A ``Goods Control Form'' can be used to record additional details, such as enumerating the contents of a package. 

\subsubsection{Packaging}
\begin{itemize}
\item When returning goods, check that everything that arrived is present, using the Goods Control Form or Goods log book.
\item Use suitable packaging.
\item Attach a clear label to the package with the client's address.
\item When the goods are classed as dangerous goods, refer to corporate dangerous goods procedures
\end{itemize}

\subsubsection{Dispatch}
The organisation’s Logistics department handles the dispatch of goods from the site.
\begin{itemize}
\item Callaghan Innovation's Courier Request Form should be completed and accompany the package.
\item Logistics must be advised that a package is ready for dispatch.
\item The client may be advised that the package is being dispatched
\item Details of the dispatch should be entered in the Goods control form or Goods log book and a copy of the courier request form stored with job records in the section file system
\end{itemize}

\subsubsection{Goods on loan from MSL}
When goods are lent to other MSL staff or to staff in the parent organisation, a record shall be kept in a section Loan Book, or a Goods Log Book, or be recorded electronically as part of the section file system. The record will show the time and date of despatch, the borrower’s name, phone number or email and location.

When MSL goods are leased, or sent outside the organisation on loan, or for repair, modification, compatibility testing, etc, a record shall be made in the Goods Log Book or be recorded electronically as part of the section file system. 
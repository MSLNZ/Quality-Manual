\section{Introduction}
The Measurement Standards Laboratory of New Zealand (MSL) is New Zealand's National Measurement Institute (NMI). MSL is designated as the verifying authority of New Zealand's national measurement standards by act of parliament. All results quoted in measurement, calibration and test reports issued by MSL are directly traceable to the national measurement standards held by the laboratory. 

New Zealand is a signatory to the international Metre Convention \cite{Metre_convention}, which establishes a world-wide system of measurement units. 
MSL plays a key role in ensuring international recognition of New Zealand's measurement system. 

MSL is a signatory of an international Mutual Recognition Agreement (MRA) \cite{MRA_1999} that provides a framework for NMIs to demonstrate the performance of measurement capabilities that support calibration and measurement services. One of the MRA requirements for mutual recognition is that each NMI implement a suitable system for ensuring quality. MSL has implemented a quality management system (QMS) that is accredited, by International Accreditation New Zealand (IANZ), against the ISO 17025 standard for testing and calibration laboratories \cite{ISO_17025}. 

For the purpose of accreditation against ISO/IEC 17025, the scope of the QMS covers all MSL activities that relate to the provision of traceable calibration and measurement services within MSL's scope of accreditation.  This includes realisation and maintenance of standards, calibration, and product testing. 

This Quality Manual also covers activities that are related to MSL's role as an NMI but are not part of ISO 17025.  Such activities include research, training and dissemination and consultancy.
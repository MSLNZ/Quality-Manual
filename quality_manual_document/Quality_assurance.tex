\section{Quality Assurance}

\subsection{Policy on the Quality Assurance of Measurement Capabilities}
\label{ss:quality_assurance_policy}
MSL sections will monitor the performance of all measurement capabilities. 

Monitoring must be planned and will include both internal and external activities. 

The results of quality assurance activities will be continuously reviewed within the section so that appropriate action can be taken to prevent unsatisfactory measurement capabilities from being used. 

Any results that suggest the performance of a measurement capability is inferior to the current accredited scope must be documented and investigated by the section. The MQC must be notified immediately about any discrepant results. The MQC will maintain a record of such instances, including a report about the corrective actions taken. 

Internal quality assurance activities shall be used to monitor the performance of measurement procedures. For example, 
\begin{itemize}
\item	regular use of certified reference materials and/or reference material and/or quality control 
material;

\item	regular use of alternative metrologically traceable instrumentation;

\item	functional checks of measuring and testing equipment;

\item	use of check-standards with control charts and/or control limits;

\item	periodic intermediate checks on measuring equipment;

\item	replicate tests or calibrations using the same or different methods;

\item	re-testing or recalibration of retained items; inter-operator comparisons;    
\end{itemize}

External quality assurance activities within the accredited scope should be planned, if possible, to occur over the usual accreditation cycle. These activities generally involve participation in international measurement comparisons with other NMIs.     

A central register of all MSL participation in international comparisons will be maintained in the QMS and will include a copy of comparison reports, which will be reviewed.  




%MSL Quality Policy Statement
%The Measurement Standards Laboratory (MSL) is New Zealand's National Metrology Institute (NMI). 
%MSL is responsible for developing and disseminating the physical measurement standards needed in 
%New Zealand and ensuring that they are accepted nationally and internationally. MSL's activities 
%support measurement capabilities that underpin New Zealand's prosperity and quality of life. 
%We aim to provide the services that our customers need and, through a process of continuous 
%improvement, to anticipate those needs and exceed our customers' expectations. We are 
%committed to providing a quality service in a safe and healthy working environment. 
%MSL's calibration and testing services are accredited by International Accreditation New Zealand 
%(IANZ), against the ISO 17025 standard for testing and calibration laboratories, using internationally 
%recognised technical experts in each area. The capability of these services is documented in the MSL 
%Scope of Accreditation schedule issued by IANZ. 
%As an NMI, MSL: 
%*	participates in international activities that ensure mutual recognition of New Zealand's and 
%other nations measurement capabilities; 
%*	provides technical measurement services for the public and private sectors; 
%*	engages in research and development activities that support the scientific and technical 
%foundation of the international measurement system; 
%*	provides scientific leadership to New Zealand's National Quality Infrastructure in the form of 
%authoritative and independent scientific advice on measurement; 
%*	provides knowledge transfer and advice for industry, Government and academia.
%MSL is committed to the level of quality expected from a national centre of excellence in metrology. 
%MSL management and staff are committed to complying with ISO 17025 and to seeking continual 
%improvement in the effectiveness of the management system. MSL staff and individuals qualified to 
%carry out test and calibration activities adhere to the policies and procedures documented in the 
%quality management system. 
%The quality of services and other activities are monitored and regularly reviewed, using information 
%collected from:
%*	client satisfaction surveys and customer feedback;
%*	MSL's performance in international measurement comparisons;
%*	international peer-reviews of calibration and measurement services;
%*	management reviews of the Quality Management System;
%*	IANZ audits of the Quality Management System;
%*	MSL audits of calibration and measurement services and the Quality Management System;
%*	MSL reviews of technical areas;
%*	health and safety audits of work areas. 

\proposedbox{
\subsection{Metrological traceability}
 \label{ss:metrological_traceability}
Both the CIPM \cite{CIPM_MRA_POLICY} and ILAC \cite{ILAC_MRA_POLICY} MRA policies insist on metrological traceability. However, the CIPM MRA is more restrictive and requires external measurements to be obtained from other NMIs or DIs. An exception is only permitted for \textit{`auxiliary influence quantities, not part of the main traceability path to the SI for a particular measurand'} if the contribution to uncertainty is `minor' (but a quantitative definition of \textit{minor} is not given). In such cases, \textit{`measurement services provided by laboratories accredited by a signatory to the ILAC Arrangement'} may be used.

\subsubsection{Traceability policy}
 \label{sss:traceability_policy}
CIPM MRA traceability policy is preferred for all MSL capabilities.

The criterion for being an `auxiliary influence' in that policy is interpreted at MSL as follows. If $u_\mathrm{cmc}$ is the nominal \textit{standard uncertainty} associated with a claimed capability, then the maximum acceptable standard uncertainty of an `auxiliary' influence factor shall not exceed $u_\mathrm{cmc} / 10$. 

\paragraph{Exceptions:}  
All MSL services provide metrological traceability, but the quality of a measurement service does not depend on how traceability is obtained. 

When a capability is published in the BIPM KCDB it must conform to CIPM MRA policy. However, when a capability only appears in MSL's IANZ scope of accreditation, it need comply with ILAC MRA policy. So, MSL may issue reports with an IANZ endorsement on the cover but no CIPM logo, to indicate that the measurement service is not a KCDB-registered CMC.

When there is good reason to provide a capability that does not comply with CIPM requirements, the justification for an exception to our policy should be recorded (for example, as part of the technical procedure). 

\paragraph{Quality assurance:}
There are no guarantees of quality. MSL has had ample experience of poor-quality calibration services provided by reputable NMIs, as well as ILAC-accredited calibration laboratories. MSL has a responsibility to assure the quality of all external measurement services that are incorporated in our measurement processes. Section 6.6 of ISO 17025 applies \cite{ISO_17025}.
 
% \paragraph{Background}
%The notion of traceability relates to the provenance of measurement error. Saying that a measurement is traceable means there has been an accurate assessment of the typical magnitude of measurement error. The measurement process must be carefully analysed and all significant sources of error must be characterised to evaluate the measurement uncertainty. Moreover, in as much as a process uses calibrations, or measured values, obtained outside the laboratory, that external data must also have accurate assessments of measurement uncertainty.  

}


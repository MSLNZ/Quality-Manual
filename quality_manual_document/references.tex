\section{References}

% The next two lines prevent 'thebibliography' from generating the title 'References' again
\begingroup
\renewcommand{\section}[2]{}%

\begin{thebibliography}{9}
\bibitem{MRA_1999} Comit\'e International des Poids et Mesures, Mutual recognition of national measurement standards and of calibration and measurement certificates issued by national metrology institutes, Paris, 14 October 1999 (\url{http://www.bipm.org/en/cipm-mra/}).
\bibitem{Metre_convention} Metre Convention (Convention du M\`etre), also known as the Treaty of the Metre, is an international treaty that was signed in Paris on 20 May 1875. The treaty set up an institute for the purpose of coordinating international metrology and for coordinating the development of the metric system. In 1960, the system of units it had established was overhauled and relaunched as the "International System of Units" (SI).
\bibitem{ISO_17025} New Zealand Standard, ISO-IEC 17025:2018, General requirements for the competence of testing and calibration laboratories, Standards New Zealand, Wellington, 2018.
\bibitem{MS_ACT_1992} \href{http://www.legislation.govt.nz/act/public/1992/0052/latest/whole.html}{Measurement Standards Act 1992}
\bibitem{NS_Regulations} \href{http://www.legislation.govt.nz/regulation/public/2019/0091/latest/whole.html}{National Standards Regulations}
\bibitem{MSL_Reporting_Guidelines}  \href{https://edi.callaghaninnovation.govt.nz/ws/msl/QMS/QM/Guidelines%20on%20Measurement%20Uncertainty.docx?Web=1}{Guidelines on Reporting and Publishing (in the EDI library)}
\end{thebibliography}

\endgroup
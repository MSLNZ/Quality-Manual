\section{Reporting and Publishing}
MSL staff report on activities in a number of ways. Of particular importance are the metrological reports (calibration and test reports) issued by MSL. These are covered in the next section. 

A variety of other works may be produced for ‘publication' outside MSL, including:
\begin{itemize}
\item Scientific journal and conference papers (peer-reviewed or not)
\item Conference posters + presentation slides
\item Callaghan Innovation technical reports
\item Trade journal articles
\item International measurement comparison reports
\item Reports/submissions to technical committees (TCs, CCs, …)
\item MSL consultancy reports
\item MSL Technical Guides
\item MSL training course slides and notes
\item MSL software manuals (for external use)
\item MSL web pages
\item Web-based videos (YouTube channel)   
\item MSL newsletter
\item Publicity material
\end{itemize}

Section~\ref{ss:msl_publications_policy} describes MSL policy for this broader category of publication.

Measurement results shall be traceable to primary measurement standards held by MSL, or other NMIs. A statement to this effect is included on all calibration and test report covers (see \cite{MSL_Reporting_Guidelines}). Any exceptions to this policy must be approved by the Chief Metrologist (~see~\S\ref{ss:msl_publications_policy}~).

\subsection{Metrological Reports}
\label{ss:metrological_reports}
Metrological reports are of high importance to MSL, so an elaborate system has been developed to ensure that they are of consistently high quality.

\subsubsection{Report types}
A metrological report will be one of the following types:
\begin{itemize}
\item Calibration Report - entitled ``Report on the Calibration of \ldots" or ``Report on the Measurement of \ldots'' 

A Calibration Report is used for IANZ classes beginning ``5'', except when the instrument or device under test is deemed to be unfit for calibration (in which case the measurement results may be issued in a ``Report on the Failure of \ldots'').

\item Test Report - entitled "Report on the Test of \ldots''  

A Test Report may be used for IANZ classes beginning ``5" or ``6"
\end{itemize} 

\subsubsection{Authority to write and review reports}
A person designated as a ‘worker’ will carry out calibration or test work and write the report; a person designated as a ‘checker’ will check the measurements and the report. The worker(s) and checker share responsibility for the report.

The last page in the body of the report shall be signed, and all other pages initialled, by the worker, the checker, and the Chief Metrologist (or delegate). 

For IANZ-endorsed reports, either the worker or checker signing the report must be an IANZ signatory for the measurement class(es) covering the measurements.

The report shall be reviewed by the checker with respect to the following: 
\begin{itemize}
\item there is sufficient documentary evidence of the measurements made;
\item the process used is described in a technical procedure, or procedures, and that each procedure used has a current validation and has been correctly interpreted (where ‘current validation' is understood to mean that the technical procedure is valid on the report date);

Note, see \S\ref{sss:technical_procedure_structure} for validity requirements when a supporting procedure provides part of a traceability chain.

\item records clearly show who did the checking and what was checked (for routine work a checklist is recommended).
\end{itemize} 

The report must be reviewed by the Chief Metrologist, or delegate, to ensure that:
\begin{itemize}
\item the report has been written and checked as required above;
\item the report is free of typographical errors or inconsistencies;
\item the correct front cover has been used, indicating whether the measurements are within the IANZ scope of accredited services and whether the measurements are within MSL’s CMCs (note also \S\ref{sss:measurement_conditions} regarding measurement conditions);
\item for IANZ-endorsed reports, at least one signatory has been approved;
\item the report date should normally be less than one month before the date that the report is signed.
\end{itemize} 

\subsubsection{Reporting measurement conditions}
\label{sss:measurement_conditions}
Sometimes, conditions of measurement are reported that involve quantities from a different technical discipline. For example, when calibrating a thermometry resistance bridge, staff with specialist knowledge in temperature measurement report a value of sensing current (an electrical quantity). Staff qualified to carry out the technical procedure (i.e., who appear in the Technical Competency Matrix for that procedure) are considered to have the expertise to assess associated measurement conditions. This is confirmed by external peer-review of signatories by IANZ. Therefore, reporting of measurement conditions is assumed to be compatible with the MSL scope of accreditation and the signatory status of the staff involved.

\subsubsection{Reporting format}
Details about the reporting format requirements are given in the Metrological Reports section of the MSL Guidelines on Reporting and Publishing \cite{MSL_Reporting_Guidelines}.

\paragraph{Statement about applicability of results}
For most calibration work, the MSL report format clearly identifies the item(s) being calibrated or tested. However, there are some situations where MSL is effectively sent an item or items that may be considered, from the point of view of the client, as a sample from a larger population (e.g., a piece of shade-cloth sent to MSL for characterisation). In all such cases, the MSL report shall clearly state that the results provided relate only to the item(s) actually measured. This is a requirement of the 17025:2018 standard \cite[clause 7.8.2.1~(l)]{ISO_17025}.      

\subsubsection{Issuing Reports and Summary Data}
Only one original for each report shall be produced, signed, and sent to the client. 

A scanned copy of this signed report shall be filed in the central file system ( see~\S\ref{ss:central_file_system} ). 

An electronic copy of the document used to create the report shall be retained in the section file system ( see~\S\ref{ss:section_file_systems} ). 
   
The client may be supplied with a copy of the scanned PDF version of the signed report, if requested.

A laminated card of the table of corrections may be supplied to the client, if required.  This card will show: the name of the instrument, the serial number of the instrument, the MSL report number and the correction table data.  The card will go through the same checking process as described above.

Clients may request calibration information in electronic format, such as an Excel spreadsheet. Such information may be provided, but only after the report has been issued. The electronic data must be checked with the same care given to written reports. Moreover, it should be made clear to the client that the written report remains the authoritative document. 

Every effort should be made to ensure that the items calibrated or tested will be returned to the client at the same time as the report. If the report is delayed, the client should be informed. If it is known in advance that the report will be issued after return of the client’s items, the client should be advised before the work begins.

\subsubsection{Withdrawal due to a reporting error or measurement error}
\label{sss:reissue_report}
If any significant error is found in the most recent report on an instrument or artefact, within five years of first issue, the report must be withdrawn.``Significant” would include a significant error of measurement or data processing, incorrect values, dates, serial numbers, etc, but not simple spelling or typing errors unless these could be misinterpreted.

In order to reissue the report, the instrument or artefact may need to be recalled so the test or calibration can be repeated. A new test date will be used if the test or calibration was repeated.

Any change of information in the reissued report shall be clearly identified and, where appropriate, the reason for change shall be given.

To withdraw a report (NB changes to the central file of reports should only be done by an MSL administrator or the Quality Manager):
\begin{enumerate}
\item Contact the client and ask for the return of the original report and also that any copies be destroyed. 

\item On receipt of the original report, stamp all pages ``{\color{red}CANCELLED}” and initial. 

\item Where paper photocopies are held by MSL, all pages should be stamped ``{\color{red}CANCELLED}” and initialled.

\item Electronic copies in the central file, and in the section file, should be cancelled:
\begin{itemize}
\item For a PDF file, add the text “{\color{red}CANCELLED}” to each page (in red text, with bold, large, font size – there are PDF reader programmes that can do this). 
\item For the central file copy, the name of the file should not be changed. Rather, the cancelled file should be saved with the same name and a note added about the cancelation in the EDI version history of the document. 
\end{itemize}

\item Produce a corrected report using the original report number.  Retain the original issue dates throughout, but add, after each issue date, the words ‘reissued on [new issue date]’, e.g. ‘Report No. Pressure/1997/155, 11 August 1997, reissued on 10 February 2000’.

\item Any change of information in the report must be clearly identified and, where appropriate, the reason for the change should be included in the report \proposed{(some advice on formatting this notification is given in the `Reissued reports' section of the Guidelines on Reporting and Publishing \cite{MSL_Reporting_Guidelines} ).}

\item Reissue the corrected report and the cancelled original report to the client.

\item Make copies of the signed corrected original and file in the central file and the section file. 
\begin{itemize}
\item The name of the central file copy should not be changed (i.e., the new copy of the report replaces the cancelled original). 
\item A note should be added in the EDI version history to indicate a reissued report.
\item The metadata tag ‘reissued’ for the EDI document should be set.
\end{itemize}

\item If appropriate, amend the technical procedure associated with the report to reduce the likelihood of a similar mistake occurring in the future.

\item If the likelihood of a similar error occurring in the future can be reduced by amending the Quality Manual, follow the Improvement Procedure.  
\end{enumerate}

\subsubsection{Replacing a report at the client's request}
When a report has been lost, a replacement may be issued. The new report must be identified as a replacement and have a new report issue date (i.e. to ensure that any original is unique). The replacement must be signed, as above, and copies stored in the central file and section job file. 
\begin{itemize}
\item The first replacement of a lost report should be labelled ``Replacement of Report \ldots". 
\begin{itemize}
\item Retain the original issue dates throughout but add the words `reissued on (new issue date)', e.g.: ``Replacement of Report No. Pressure/1997/155, 11 August 1997, reissued on 10 February 2000".
\item Any subsequent replacements should be labelled ``Second Replacement of Report \ldots", etc.
\end{itemize}

\item Where an original signatory is not available, a person appointed to carry out the test and calibration work, or checking, in the designated field may sign the report per persona (p.p.).  The name of the original signatory must remain on the report.

\item Reissue the report to the client.

\item Make copies of the report and file in the central file and the section file.
\begin{itemize}
\item The name of the central file copy should not be changed (i.e., the new copy of the report replaces the original). 
\item A note should be added in the EDI version history to indicate a replaced report.
\item The metadata tag ‘reissued’ for the EDI document should be set.
\end{itemize}
\end{itemize}

Note, the client, on request, may be supplied with a copy of the original report, in which case the report is not considered ``reissued". The client should be advised that the copy is not authoritative and, for example, may not be acceptable in court.

\subsubsection{Replacing a report issued with an incorrect report number}
\proposedbox{
A replacement may be issued if a report has been issued with an incorrect report number. 

The first steps of the procedure in \S\ref{sss:reissue_report} for re-issuing a report should be followed, namely:

\begin{enumerate}
\item Contact the client and ask for the return of the original report and also that any copies be destroyed. 

\item On receipt of the original report, stamp all pages ``{\color{red}CANCELLED}” and initial. 

\item Where paper photocopies are held by MSL, all pages should be stamped ``{\color{red}CANCELLED}” and initialled.

\item Electronic copies in the central file, and in the section file, should be cancelled:
\begin{itemize}
\item For a PDF file, add the text “{\color{red}CANCELLED}” to each page (in red text, with bold, large, font size – there are PDF reader programmes that can do this). 
\item For the central file copy, the name of the file should not be changed. Rather, the cancelled file should be saved using the same name and a note added about the cancellation to the EDI version history of the document. 
\end{itemize}
\end{enumerate}

The rest of the procedure is as follows: 
\begin{enumerate}
\setcounter{enumi}{4}
\item The replacement report should be labelled with the correct (new) report number, followed by a line begining ``Replacement of Report \ldots", which describes the previously issued report. For example, 
\begin{quote}
Report No.\ Humidity/2020/428, 3 March 2020\\
 Replacement of Report No.\ Humidity/2020/248, 16 February 2020
\end{quote}

\item The replacement report must be signed.
\begin{itemize}
\item Where an original signatory is not available, a person appointed to carry out the test and calibration work, or checking, in the designated field may sign the report per persona (p.p.).  The name of the original signatory must remain on the report.
\end{itemize}

\item Make copies of the report and file in the central file and the section file.
\begin{itemize}
\item The name of the central file copy should be the new file name. 
\item The metadata tag ‘reissued’ for the EDI document should be set.
\item A note should be added in the EDI version history to indicate a replaced report.
\end{itemize}
\item Reissue the report.

\end{enumerate}


}

\subsection{MSL Publications policy}
\label{ss:msl_publications_policy}

This section describes a general policy for MSL publications (other than calibration or test reports). 

Publication is very important for disseminating information held by MSL. There are many different methods of publishing, so a careful choice should be made of the best way of communicating with the intended audience. In general, technical sections should plan such dissemination. It may be desirable to use several channels; for example, after substantial scientific work has been published in peer-reviewed journals, or presented at specialised conferences, further dissemination can effectively use technical guides, training sessions and online material. 

The intent of this policy is to maintain a high standard of publications. This is to the obvious benefit of MSL’s reputation but also, and perhaps more importantly, it is critical to the effectiveness of dissemination. 

Publications should be reviewed before release, but the level of checking will not usually be as thorough as for calibration or test reports. The importance attached to reviewing should take into account the potential for harm that could occur if poor quality material is made public, as well as the ease with which any errors might be corrected if discovered after release of the material (for instance, we can correct mistakes in a document distributed via our website, but mistakes in a journal article are more difficult to correct).  

As a general principle, it is intended that
\begin{itemize}
\item  All material be reviewed before publication
\item  Measurement results will be traceable to primary measurement standards held by 
\end{itemize}
MSL, or other NMIs; when this is not the case, a statement about the lack of traceability shall be included (see notes below)
\begin{itemize}
\item  No measurement results, traceable or untraceable, will be released without the approval of the Chief Metrologist (or delegate) 
\item  Material for review will be made available in a common repository (~see~\S\ref{sss:publications_repository}~)
\item  A record of reviews will be kept with the material in the repository 
\end{itemize}
Notes: 
\begin{itemize}
\item  The record of a review (in the repository) may be quite detailed or contain sufficient information to locate such detail elsewhere (lab books, spreadsheet on the I-drive, etc), or it may simply state that the work has been viewed and is of publication quality.  
\item  Given the many different types of ‘publication’, a single repository is unlikely to cater for all. For example, modern websites usually have content management tools that are better suited to managing an approval process before publication.  
\item  Some material, for example the slides for a technical talk, may be produced too close to the due date to allow time for review. The material should nevertheless be placed in the repository as soon as possible. 
\item  Vigilance is needed to avoid unintentional publication of untraceable measurements. A notable example was the time-of-day widget initially displayed on our public website. Note, all traceable measurements will have a stated uncertainty and documentation that can be independently checked as evidence of traceability and accuracy.
\item  Measurements may rarely be published without traceability, however, a clear statement about the lack of traceability should accompany the results (see the section ‘Informal reporting of measurements’ in the Guidelines on Reporting and Publishing \cite{MSL_Reporting_Guidelines}). In some cases, it may be possible to offer useful advice. For instance, 
\begin{quote}\textit{
The time of day displayed here is affected by unpredictable transmission delays between MSL and the display device. It should not be used as a standard. Advice on how to obtain accurate time services from MSL is available here <link to guides>”.
}\end{quote} 
\end{itemize}

\subsubsection{Responsibilities}
\begin{itemize}
\item  One of the MSL authors will act as ‘lead’ in the publications process
\item  It is the lead author’s responsibility to make a final copy of the material available in the repository and to advise the Quality Manager and the Team Manager.
\item  It is the Quality Manager’s responsibility to review the material, or to delegate this review.
\item  It is the responsibility of the lead author’s Team Manager to consider potential issues related to confidentiality, impartiality or intellectual property.
\end{itemize}

Note: Authors may nominate reviewers to the Quality Manager.

\subsubsection{MSL publications repository}
 \label{sss:publications_repository}
The MSL Publications library, of the ‘Measurement Standards Laboratory’ site on EDI, is the common repository for publications.  The following types of document can be classified: Peer-reviewed journal, Conference proceedings, Book chapter, Book, MSL Technical Guide, Technical Report, MSL Training Course, Presentation, Poster, and MSL Consultancy Report. 

\subsubsection{MSL publications clearance form}
An MSL publication clearance form template can be used in the repository. The form is available as one of the ‘new’ document types that can be created from inside the docset (‘bucket’) associated with the publication.

The publication clearance form can be used to create a review record; the versioning system on EDI will keep track of the different individuals who create the record (e.g., a Team Manager can save comments about IP before or after a technical review by someone in the team). Extra rows can be added to the form for authors or reviewers (a table structure is used in the form).

This form is provided for convenience; other types of document could also be used to keep a record of review. The form is designed for use with technical scientific papers and reports (i.e., traditional types of scientific publication).  

\subsubsection{Minimum standards}
All material must be presented clearly, with consideration given to the intended audience.

Technical material will be reviewed for appropriate scientific rigour and technical soundness. 

Material will be of appropriate visual appearance and graphic standards (~see~\cite{MSL_Reporting_Guidelines}~). Sometimes existing documents can serve as a guide, and templates for some types of publication are available.  

When the work will be published by an external body, such as articles in scientific journals and trade journals, authors should try to check publisher proofs. This is normal for scientific journals, but it is also important to do this for trade journals when mathematical symbols, units and equations are being used, because editors and journal staff are often unfamiliar with these elements.  

\subsubsection{Check lists}
The following lists may be helpful during reviews
\paragraph{Scientific papers (journals and conference proceedings)}
\begin{itemize}
\item  Review for technical correctness.
\item  Review for clarity of expression and the consistent use of mathematical language and symbols.  If available, follow a publisher’s guidelines, otherwise \cite{MSL_Reporting_Guidelines} contains advice.
\item  Review tables and figures for consistency of appearance and clarity. Follow the publisher’s style guide, if available, otherwise see \cite{MSL_Reporting_Guidelines}.
\item  Check bibliographic references. Follow the publisher’s style guide, if available.
\item  Is any part of the work covered by third party agreements? If so, is the material for publication consistent with those agreements? 
\item  Are any IP protection steps needed before release?
\item  Have all contributors been acknowledged
\item  Are all authors satisfied with the final draft?
\item  Has funding been acknowledged?
\end{itemize}

Sometimes, a last-minute rush before a conference puts pressure on the time available for review. Nevertheless, many of the points above can be checked before the final version of a text is ready, in particular issues related to IP and third-party agreements can be cleared with the Team Manager.

\paragraph{Trade journals}
\begin{itemize}
\item  Follow the points listed above for scientific papers.
\item  Is the writing at a suitable level for the intended audience?
\item  Remember to insist on a review of proofs before publication.
 Callaghan Innovation technical reports
\item  Follow the points listed above for scientific papers.
\item  Consider including a Creative Commons licence (~see \cite{MSL_Reporting_Guidelines}~).
\item  Has a report number been allocated by the Library? Note that the Library keeps its own archival copy in an EDI bucket that is created at the same time as a report number is issued. A PDF copy of the final version of the report should be stored there and the status attribute of the bucket modified accordingly. Please consult Library staff or the Quality Manager if you need assistance with this process.
\end{itemize} 

\paragraph{Technical Guides}

\begin{itemize}
\item MSL Technical Guides should be readily identifiable as a product of MSL. They should clearly display the MSL logo on the front page and include generic contact information for MSL (email, www). Templates for Technical Guides are kept with the current guide documents on the I-drive (look for: 
\verb|MSL\Private\Technical Guides|).
\item  Make sure the guide has been given an MSL technical guide number
\item  The current MSL email contact is info@measurement.govt.nz
\item  The current MSL website URL is www.measurement.govt.nz
\item   A version number and date of publication should appear on the front page
\item  The person responsible for the Technical Guide should be identified, but individual contact details should not be given
\item  Review for technical correctness.
\item  Review for clarity of expression and the consistent use of mathematical language and symbols (~see~\cite{MSL_Reporting_Guidelines} for useful advice~).
\item  Review tables and figures for consistency of appearance and clarity (~see~\cite{MSL_Reporting_Guidelines}~).
\item  Check bibliographic references. 
\item  Include a Creative Commons licence (~see~\cite{MSL_Reporting_Guidelines}~).
 Training course slides and notes
In most cases PowerPoint slides will be combined with slide notes to produce handouts for attendees (although some courses provide a separate booklet). 
\item  Make sure that the cover page is easily identifiable as a product of MSL (name, logo, etc)
\item  Make sure there is a description of the course structure and learning objectives
\item  Make sure that thought has been given to navigating the notes (e.g., contents, page numbering, section breaks, etc) 
\item  Review material for technical correctness.
\item  Review material for suitability and for continuity of ideas (is the message clear; are things being introduced in the right order; are there unnecessary bits that could distract)
\item  Review for clarity of expression and consistent use of mathematical language and symbols.  
\item  Check any bibliographic references. 
\item  Consider including a Creative Commons licence (~see \cite{MSL_Reporting_Guidelines}~).
 Posters and presentations
\item  Follow the points listed above for scientific papers.
\item  Make sure that the work is easily identifiable as a product of MSL (use of name and logo, contact details, web URL, etc). Templates are available or recycle an existing document.
\end{itemize} 

Often, posters and presentations are not reviewed, either because there is not enough time before the due date, or because a review is not considered sufficiently important to take up the time of another staff member. Nevertheless, the final work should still be placed in the repository as soon as possible. 

\subsubsection{MSL consultancy reports}
MSL consultancy reports are written in the form of a letter, unless specific format is requested by or deemed appropriate for the client.
A consultancy report must NOT be used to present the results of measurements. Any such results shall be contained in a separate test report to which reference may be made in the consultancy report.

A report must be checked by the Quality Manager or nominee primarily for soundness of approach to the problem and clarity of exposition. The Quality Manager shall also check whether: 
\begin{itemize}
\item  The author is qualified to write the report
\item  The opinions expressed are fundamentally sound
\item  The ramifications of the report (legal, financial etc) require that it be approved at a higher level (e.g. Group Manager)
\end{itemize}
A copy, signed by both the author and the checker, shall be deposited in the publications repository (~see~\S\ref{sss:publications_repository}~). 

\subsubsection{Reporting on international measurement comparisons}
Reports on MSL participation in international comparisons should be made in the form of a calibration report, unless another format is dictated by the requirements of the comparison. In any case, the report will be checked as thoroughly as for calibration reports.

\subsubsection{Acknowledgement of national standards funding} 
Suitable wording to acknowledge the funding of national standards is either:
\begin{quote}
\textit{This work was funded by the New Zealand Government}
\end{quote}
or
\begin{quote}
\textit{This work was funded in part by the New Zealand Government}
\end{quote}

Such wording should normally be included in an Acknowledgements section at the end of works appearing in external publications, such as scientific journals and conference proceedings.
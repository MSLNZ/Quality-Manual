\section{Improvements}
\label{s:improvements}
MSL will continually improve the effectiveness of its quality management system. The laboratory shall take a proactive approach: rather than merely checking for conformity, we will actively seek to identify potential risks and opportunities for improvement.

There are two mechanisms for formally identifying issues with the quality management system, or opportunities to improve it. One is the MSL ‘Improvement Request’ (IRF) process, the other is the Risks and Opportunities register. 
Any member of MSL may raise an improvement request or add an item to the Risks and Opportunities register. 

The Risks and Opportunities register was introduced by 17025:2017. We generally use it as a way to flag things that would previously have been classified as “preventative actions” under 17025:2005. The register records potential issues or opportunities at an early stage, allowing them to be managed. Entries may grow (as an opportunity or potential issue is realised) or disappear. Entries that grow will usually be promoted to an IRF.

The IRF process is used when non-conformities are identified (corrective actions) and when substantial opportunities for improvement arise, or potential future instances of non-conformity arise (formerly called “preventative actions”). The IRF process allows detailed documentation to be collected about an issue, it’s context, discussions and actions taken. 
Note that risks to business continuity, health and safety, conflicts of interest are managed by other corporate policies. 

\subsection{Risks and Opportunities}
The laboratory needs to review risks and opportunities to the quality management system regularly, just as risks and opportunities related to Health and Safety, etc, are reviewed. \\
\\
The MSL implementation of Risk and Opportunities uses the Callaghan Innovation Risk Management Framework and is implemented as the JIRA Smart Risk project (\url{https://callaghaninnovation.atlassian.net/jira/software/c/projects/CISR/boards/122}). \\
Instructions for creating a new risk are available at \url{https://sites.google.com/callaghaninnovation.govt.nz/msl-qms/home/risk-and-opportunities-register}

Each Risk or Opportunity entered records the nature of a risk or opportunity, assigns a Reporter, Assignee, Risk Owner, possible contingencies (planning) should the issue arise, and the most recent review taken (with the date). \\
\\
Note: The Reporter is set by default as the person logged in while raising the risk in JIRA. Assignee and Risk Owner are initially set to the Reporter's manager.  The appropriate people to be Assignee and Risk Owner are reassigned as appropriate when the risk is initially reviewed by the management team.

This register shall be reviewed regularly by the MQC and by the MSL Leadership group. It is the responsibility of MSL management to ensure that staff are consulted regularly. 

The Risks and Opportunities in JIRA act as a light-weight register; as soon as an item in the register becomes substantial, an IRF will usually be raised.

\subsection{Improvement requests (IRF)}
\label{ss:improvement_requests}
Improvement requests may arise in relation to instances of non-conformity, as well as when an opportunity for improvement or a potential future source of non-conformity is identified. 

Improvement requests may be generated by different types of activity, including internal and external audits, analysis of data, management reviews, and customer feedback. 

A corrective action must be undertaken when a quality problem is identified. This process is documented and managed within the quality system using an IRF. A complaint (or someone simply drawing attention to a quality problem) originating from outside MSL should also trigger an IRF (~see~\S\ref{ss:complaints}~).

It is important to recognise that a corrective action is intended to address and preferably eliminate the underlying cause of a problem, not merely correct a detected non-conformity. There will be occasions when the appropriate response to a non-conformity is just a simple correction, but more generally a corrective action should determine the root cause of a non-conformity and eliminate that cause. The effectiveness of a corrective action will be monitored by the MQC to ensure that it has been successful.  An improvement request will be considered closed when the MQC is satisfied that the action has been effective.

Whenever a potential non-conformity or needed improvement is identified, this finding can also be raised as an IRF. The IRF should then be investigated to determine a suitable preventative action to avoid or mitigate an occurrence of a non-conformity or to improve the system. While an opportunity to strengthen the quality system may be identified, the risk of not carrying out a preventative action should be weighed against the potential benefits. If a decision is made not to proceed with a preventative action, the justification for this decision needs to be recorded in the IRF.

The effectiveness of a preventative action will be monitored by the MQC. The MQC will decide on the most appropriate monitoring process and the IRF will remain open until MQC is satisfied that the action has been effective. 

\subsubsection{IRF procedure}
\label{sss:irf_procedure}
Improvement requests are currently managed using the Improvement Request folder within the central file system (~see~\ref{ss:central_file_system}~). 

Instructions on how to raise a new IRF are provided on the MSL Quality Management System help site, or for convenience \\
\url{https://sites.google.com/callaghaninnovation.govt.nz/msl-qms/home/improvement-requests}.

The person raising the IRF should fill in the first part of the Improvement Request form, including a description of any immediate action that has been taken. \\
\\
Once the initial IRF form has been submitted to the Quality Manager, the Quality Manager shall
\begin{itemize} 
\item assign a number to the IRF using the IRF register
\item create a new subfolder, identified by the new IRF number and description, within the Improvement Requests folder in the AODocs Quality Management System site
\item move the original submitted IRF form to the new folder
\item notify the submitter of the IRF of the IRF number and folder location documents
\item consider whether any immediate action needs taken and initiate any suitable action
\end{itemize}


The Quality manager and MQC will review each improvement request. They will consider whether an underlying cause has been identified and if not, they will take necessary measures to do so. After identifying the root cause, an appropriate corrective or preventative action will be determined, and an appropriate monitoring process will be put in place to evaluate the effectiveness of the action.  The MQC will be responsible for determining when to stop monitoring.

The status of IRFs in the central file system progresses through four states:
\begin{center}
{\renewcommand*{\arraystretch}{1.4}
\begin{tabular}{p{14.07em}p{25em}}
	\rowcolor[rgb]{ 0,  0,  0} 
	\textcolor[rgb]{ 1,  1,  1}{\textbf{Status}} & 
	\textcolor[rgb]{ 1,  1,  1}{\textbf{Comment}} \\
New & The IRF has been raised, but not yet reviewed by the MQC. \\ 
Active & The MQC has considered the IRF. A suitable action will be chosen and carried out during this phase. \\ 
Monitoring & The effectiveness of the action will be monitored during this phase. \\ 
Closed & The improvement request is closed; the action is considered effective by the MQC. \\ 
\hline 
\end{tabular} 
}
\end{center}

\subsection{Complaints}
\label{ss:complaints}
Complaints provide valuable feedback on the operation of a quality system. The issues raised should be investigated in the same manner as any other quality problem. 

A complaint may simply consist of notification, by someone external to MSL, of a quality issue. For example, something wrong with the MSL Talking Clock. 

The issue should be verified, and the complaint acknowledged.
\begin{itemize}
\item If the issue is minor and can be dealt with immediately, then do so;
\item If the complaint is substantial but straight forward, try to negotiate a solution with the complainant;
\item If the complaint is too complicated, inform the Team Manager or MSL Director
\end{itemize}

In all cases of complaint, an improvement request (IRF) must be raised. The responsibility for this lies with the person receiving the complaint but may be passed to another appropriate member of MSL.

The MQC will review the IRF (~see~\ref{sss:irf_procedure}~), to ensure that the root cause is identified and that an appropriate corrective action is carried out.  
The MQC will also ensure that a complaint has been acknowledged and that, where appropriate, the complainant is updated on progress until the issue is resolved (e.g., a negotiated solution has been actioned). 

Communication with the complainant shall be made, or reviewed and approved, by staff not directly involved in the activity in question.

When the issue has been resolved, the MQC will recommend a suitable closing action. In most cases, the Team Manager or Director will contact the complainant to ensure that the outcome is satisfactory. 

A record of all complaints should be maintained by MSL sections. Records of positive feedback and client suggestions for improvements should also be kept.
(note, this `Complaints Section' of the Quality Manual can be made available to any interested party.)

\subsection{Assessing Client Satisfaction}
\label{ss:client_satisfaction}
\subsubsection{Call backs}
MSL contacts a fraction of clients shortly after work is complete to assess satisfaction.

The intention of the call-back process is to contact roughly 25\% of clients and to record their feedback. The call should be made within roughly a month of the work being done.

A selection of clients can be achieved by a running lottery of jobs completed, or by simply choosing clients. Our policy is to avoid calling a client twice within a 24-month period.

The person making the call-back will keep a record of the conversation, which is filed centrally in the Client Callbacks library. 

Callback records are reviewed by the MQC and a register of call-backs is maintained in the MSL Quality Management System AODocs library.

\subsubsection{Client survey}
A formal client survey will be carried out occasionally. The timing of surveys will be determined by MSL management.
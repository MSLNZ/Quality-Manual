\section{Competencies, Training and Professional Development}
\label{s:competencies_professional_development}
MSL recognises that its most valuable assets are the technical competencies of staff.  The performance and reputation of MSL as a respected National Metrology Institute is founded on the calibre and achievements of its members. 

MSL also recognises that to acquire, develop and maintain competencies, staff will become involved in a variety of training and professional development activities. Such as (but not limited to):
\begin{itemize}
\item providing or receiving mentoring;
\item participating in technical meetings, such as scientific conferences, seminars and specialist metrology workshops;
\item secondments to other institutions, to acquire knowledge or skills not available at MSL;
\item formal study, such as tertiary programmes and professional development courses;
\item conducting scientific research.
\end{itemize}
\subsection{Training and Professional Development Planning}
MSL will ensure that staff have the technical competencies required to carry out all relevant present and anticipated tasks. 

Training and development needs will be identified by Team Managers in consultation with each staff member and development plans will be prepared and reviewed at regular intervals. Such planning will consider technical training needs and any requirements to ensure on-going technical competence. 

A record of all training and professional development activities will be maintained for each individual (~see~\S\ref{sss:training_records}~). Among other things, these records support formal recognition of technical competencies. They must therefore contain evidence that required standards of competency have been achieved. This evidence will be validated by a mentor (e.g., indicated by initialling relevant documents in the training record, lab notebooks, etc) who will then prepare a summary of evidence for the Quality Manager, for final approval.

Sometimes, technical competencies are recognised based on the prior experience of an individual. In such cases, a justification for the competency claimed will be filed in the staff member's training record and validated by the Chief Metrologist, before final approval by the Quality Manager. 

\subsection{Mentoring}
\label{ss:mentoring}
Mentoring is an important aspect of professional development at MSL. It is the primary means for transfer of expertise from senior to junior metrologists. It is also the primary method for evaluating and validating technical competence.

The Team Manager, in consultation with the Quality Manager or Chief Metrologist, will assign suitable mentor(s) to a staff member. The purpose of mentoring, and an expected time frame to achieve the goals set, will be noted and filed as a note or memo in the training records of both the staff member and the mentor. Note, an individual may have several mentors concurrently.

\subsubsection{Working under supervision}
While someone is acquiring experience in a particular competency, they will generally work under the supervision of a mentor. In such cases, the mentor assumes responsibility for the work outputs of the mentee. In particular, if a staff member is working under supervision on a commercial job, careful records must be kept (e.g., in the lab book or job file) to show that the mentor has supervised the work and accepts responsibility for the mentee's activities.

\subsection{Technical Competency Classes}
There is a set of five competency classes used to classify competencies. Two classes relate to the development and maintenance of technical procedures, two relate to the execution of test and calibration work and one relates to the authority to issue IANZ-endorsed reports. 

\paragraph{Author}
An author can develop new technical procedures (and other technical material, e.g.: technical guides).

Note, although ‘author' suggests the preparation of written material, the core competency is development of technical procedures, which includes appropriate documentation.

\paragraph{Reviewer}
A reviewer can review material prepared by an ‘author'. 

\paragraph{Worker}
A worker is considered competent to execute a particular technical procedure for test or calibration work, and to prepare the test or calibration report.

\paragraph{Checker}
A checker can review work done by a ‘worker'.

\paragraph{Key Technical Person}
A person with authority to issue IANZ-endorsed test, calibration or measurement reports.

\subsubsection{Competency requirements for different classes}
The particular competencies expected of an author, reviewer, worker and checker vary across the different technical areas in MSL. In some sections, checking a report amounts to little more than verifying that numbers have been transposed without error, while in others checking requires a detailed understanding of the measurement principles. 

\paragraph{Core requirements} \mbox{}\\
Technical staff in MSL will all have an appropriate set of skills, experience and knowledge in a technical area, or a related field, including:
\begin{itemize}
\item science, engineering, mathematics and statistics;
\item instrumentation and measurement techniques;
\item software development, testing and validation;
\item traceability and ISO 17025 requirements.
\end{itemize}
 
\paragraph{Author} \mbox{}\\
An author is expected to have appropriate skills, experience and knowledge in the particular competency area.

Appropriate training may involve the development of technical procedures, including written material, under supervision.

An author in training is usually mentored by someone with ‘reviewer' competency in the same, or a closely related area. A person with ‘author' competency in the same area could act as mentor.
 
\paragraph{Reviewer} \mbox{}\\
A reviewer is expected to have an appropriate set of skills, experience and knowledge in a particular technical area to adequately review technical material prepared by an ‘author'.  In addition, a reviewer is generally expected to have sufficient knowledge of the technical processes involved in the particular competency to be able to mentor all other roles.

A reviewer in training is usually someone with several years of experience in the same, or a closely related area; a mentor for a reviewer will usually be someone with ‘reviewer' competency in the same, or a closely related, area. 

Appropriate training may involve the critical review of technical material under supervision.
 
\paragraph{Worker} \mbox{}\\
A worker in a particular technical area is expected to 
\begin{itemize}
\item understand the technical basis of the relevant technical procedures, specifically an understanding of the key influence quantities and be able to recognise when something has gone wrong with a measurement;
\item have successfully operated the relevant technical procedures an appropriate number of times under minimal supervision.
\end{itemize}
A worker in training is usually mentored by someone with ‘reviewer’ competency in the same area. A ‘worker’ in the same competency could act as mentor.

Appropriate training will generally involve working under supervision but may also include the development of skills in operating instrumentation and software systems, an understanding of relevant measurement uncertainty concepts and knowledge about ISO 17025.
 
A checker in a particular technical area is expected to have an appropriate set of skills, experience and knowledge to be able to adequately review the output of a ‘worker'.

A checker in training is usually mentored by someone with ‘reviewer' competency in the same area. A person with ‘checker' competency in the same area could act as mentor.

Appropriate training may involve reviewing test or calibration work and reports under supervision. An appropriate understanding of relevant instrumentation and software systems, measurement uncertainty concepts and knowledge about ISO 17025 is also required.

\paragraph{Key technical person} \mbox{}\\
Someone with Key Technical Person (KTP) status is authorised to issue measurement, calibration and test reports endorsed with the IANZ accreditation logo. In doing so, they take full responsibility for the validity of the work.

KTP are experienced staff. They shall hold at least three of the competency classes: Worker, Checker, Author and Reviewer, for the technical procedure in which KTP competency is recognised. They shall also be familiar with MSL’s quality management system, with the IANZ rules pertaining to the use of IANZ endorsement and with the requirements of the ISO/IEC 17025 standard.  

KTP competency will be approved by the Quality Manager. 

Applications for KTP competency must be in writing and shall include a brief CV for IANZ (see figure~\ref{f:ktp_cv}). Evidence supporting the competency shall be available (usually individual training records) when the application is made.

An example showing the format for an IANZ CV follows:

\begin{figure}[ht]
\begin{center}
\begin{tabular}{|p{3.5cm}|p{10cm}|}
\hline 
\rule[-1ex]{0pt}{2.5ex} Name & Blair Hall \\ 
\hline 
\rule[-1ex]{0pt}{2.5ex} Position & Principal research scientist \\ 
\hline 
\rule[-1ex]{0pt}{2.5ex} MSL technical section & RF and microwave \\ 
\hline 
\rule[-1ex]{0pt}{2.5ex} IANZ Classes & 5.93(b), 5.95 \\ 
\hline 
\rule[-1ex]{0pt}{2.5ex} \makecell[tl]{ Qualification\\ (highest relevant)} & Doctorate \\ 
\hline 
\rule[-1ex]{0pt}{2.5ex} Relevant experience & \makecell[tl]{
• Worked in RF and microwave at MSL since 1998;\\
• Designed and developed all current MSL RF and microwave\\
 standards and services;\\
• Attended ISO/IEC 17025 training by NZQC\\ 
(for both 2005 and 2017 standards);\\
• Written more than 100 reports and research papers in the field\\ 
of radio and microwave frequency metrology;\\
• Provided training courses in RF and microwave metrology\\ 
in NZ and abroad;\\
• Acted as technical expert for IANZ audits of NZ calibration\\ 
laboratories since 2009;\\
• Participated in an international measurement comparison\\ 
for calibration factor of RF power meters (APMP.EM.RF-K8.CL).} \\ 
\hline 
\end{tabular} 
\end{center}
\caption{An example of the type of CV required by IANZ for a new Key Technical Person.}
\label{f:ktp_cv}
\end{figure}

\subsection{The Technical Competency Matrix}
A register of technical competencies, called the Technical Competency Matrix (~TCM -- see~\ref{ss:central_file_system}~), is maintained. 

The Quality Manager may add or delete TCM entries. 

Competency, once established, is reviewed during revalidation of technical procedures (~see~\S\ref{sss:tp_validation}~).  The review team shall ensure that each qualified individual (worker, author, etc) has maintained competency and is aware of any changes to the procedure. This will be noted on the validation report. 

\subsubsection{Procedure for adding a technical competency}
\label{sss:tcm_procedure}
A competency may be added:
\begin{itemize}
\item The mentor will provide the Quality Manager with a summary of the individual's training and experience in support of the competency claim ( in some cases the Chief Metrologist, rather than a mentor, will validate evidence in support of a competency and present it to the Quality Manager, see~\S\ref{ss:mentoring} ). Note, there is a Training Record Template on EDI in the Quality Manual Library, but any document will do.

When requesting KTP competency, a mentor is not required. The applicant shall apply to the Quality Manager directly and provide the necessary evidence of competency.

\item The Quality Manager can authorise the new competency by signing, if a paper document has been provided, or clearly indicating acceptance by some other means for electronic documents 

\item One copy is returned to the individual concerned, to be filed in their training record

\item Another copy is provided to the EDI site administrator, who will make a new entry in the TCM (with status ‘pending') 

\item The Quality Manager will verify the TCM change and set the status to ‘approved' in the TCM (a record of this is retained by the EDI versioning system)

\item A copy of the documents presented to the Quality Manager in support of the new competency will be saved in the Competency Records library on EDI.
\end{itemize}

\subsubsection{Procedure for deleting a technical competency}

To remove a competency from the TCM, an application may be made to the Quality Manager by the Team Manager or Chief Metrologist at any time.

\begin{itemize}
\item the Quality Manager will consider the application and authorise the requested change (by signing, if the application is on paper, or clearly indicating acceptance by some other means for electronic documents).

\item one copy will be returned to the individual concerned, to be filed in their training record

\item another copy will be provided to the EDI site administrator, who will delete the TCM entry and set the status to `pending' 

\item the Quality Manager will verify the change and set the TCM item status to ‘approved' (recorded by the EDI version control system)

\item a copy of the documents presented to the Quality Manager in support of the change will be filed for future reference in the Competency Records library on EDI.
\end{itemize}

The Quality Manager may delete complete rows in the TCM when a person no longer works with MSL. The Quality Manager shall consult with the Team Manager concerned, or with senior members of the team, before removing the rows (sometimes staff agree to remain available, in special circumstances, for a period of time after leaving MSL).

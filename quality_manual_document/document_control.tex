\section{Documents and document control}
\label{s:documents_and_document_control}
\subsection{File systems}
The QMS file system comprises a central file system and a number of section file systems. Central files contain documents and information relating to MSL as a whole, including the functions of the Quality Manager and Quality Council.  Section file systems contain documents more specifically related to the activities and capabilities of the section.

\subsection{Central file system}
\label{ss:central_file_system}
The central file system is located on the Callaghan Innovation electronic document and records management system (EDRMS, fondly referred to as EDI). In the past, the central file system was paper-based, with files located in the administration office of MSL and in the office of the Quality Manager. The transition to an electronic file system began in 2016. The older paper file system is being operated concurrently with the new electronic file system (paper files are in the storage room on the 4th floor of the Library).

The technology used in the new central file system is Microsoft Sharepoint 2013. There is a ‘Measurement Standards Laboratory’ Sharepoint site on EDI; the central files are held on a sub-site named ‘Quality Management System’ (QMS). 
Sharepoint backs up documents and provides version control, which ensures that the current version of QMS documents are easily identifiable and available uniformly throughout MSL.

Access to documents on the QMS site (and the Jobs Register on the MSL site) is limited to staff working at MSL and certain key members of the organisation who need access. 

The QMS site uses Sharepoint libraries, lists and docsets to manage documents and information. Sharepoint lists can present information in a tabular format (rather like an Excel worksheet). Loosely speaking, a Sharepoint library is a collection of documents. A library may also contain collections of docsets, which themselves may contain documents (rather like a Windows folder). The QMS site uses libraries of docsets to create two-tiered collections of documents (rather like a folder that contains other folders). 

The following table identifies the main libraries and lists on the QMS site and summarises the type of information stored in each.

{\renewcommand*{\arraystretch}{1.4}
\begin{longtable}{p{14.07em}p{25em}}
	\rowcolor[rgb]{ 0,  0,  0} 
	\textcolor[rgb]{ 1,  1,  1}{\textbf{Name}} & 
	\textcolor[rgb]{ 1,  1,  1}{\textbf{Contains}} \\

\textbf{Quality Manual} & A library containing this document, a number of guidance documents that supplement this manual, and some forms used by the QMS \\

\textbf{Improvement Requests} & A library of docsets. Each docset contains documents associated with a particular Improvement Request. \\

\textbf{Risks and Opportunities} & A library containing an Excel spreadsheet used for recording risks and opportunities to the Quality Management System \\

\textbf{Signing Delegations} & A list of the people with delegated authority to sign for the Chief Metrologist and for the Quality Manager \\

\textbf{Technical Competencies} & A list representing the Technical Competency Matrix: the official record of MSL technical competencies. \\

\textbf{\makecell[tl]{Validated Procedures\\ Register}} & A list of MSL Technical Procedures with information about their validation status. \\

\textbf{Client Callbacks} & A library of documents related to call to clients about work done recently by MSL \\

\textbf{Competency Records} & A library of docsets. Each docset contains documents presented to the Quality Manager as evidence of an individual's competency \\

\textbf{Correspondence} & A library of important correspondence related to the quality system. Mostly communications to and from the Quality Manager, IANZ, Director or Chief Metrologist relating to the accredited laboratory. \\

\textbf{Internal Audits} & A library of docsets. Each docset contains documents associated with one round of internal audits. \\

\textbf{MQC documents} & A library of MSL Quality Council working documents \\

\textbf{\makecell[tl]{Technical Procedure\\ Records}} & A library of docsets. Each docset contains a copy of the validation cover sheet and other supporting validation documents, such as the change history, for a validated technical procedure. \\

\textbf{Calibration reports} & A library containing copies of issued calibration and test reports \\

\textbf{CMCs and IANZ Scope} & A library containing the CMC documents (from the BIPM database) and IANZ scopes of accreditation for MSL \\

\textbf{International comparisons} & A library of docsets containing documents related to the participation of MSL in international measurement comparisons \\

\bottomrule
\end{longtable}
}%

\subsubsection{The MSL Quality Manual}
The documents listed in this table form the MSL Quality Manual. 
{\renewcommand*{\arraystretch}{1.4}
\begin{longtable}{p{25em}}
	\rowcolor[rgb]{ 0,  0,  0} 
	\textcolor[rgb]{ 1,  1,  1}{\textbf{Full Name}} \\
	
Quality Policy Statement (signed) \\

Quality Manual (this document) \\

Guidelines on Measurement Uncertainty \\

Guidelines on Reporting and Publishing \\
\bottomrule
\end{longtable}
}%

Note, `Guidelines' documents are considered appendices of the Quality Manual and contain detailed information on various topics. 

\paragraph{Amendments to the Quality Manual}
Any MSL Member, or staff member of the MSL group, may request a change to the Quality Manual. 

Requests for additions, changes, and deletions to the MSL Quality Manual shall be made using the improvement request procedure ( see 10.2.1 ).
It is the responsibility of the MQC to decide what changes shall be made.

\subsection{Section file systems}
\label{ss:section_file_systems}
Sections maintain a file system containing technical records, such as raw and processed data and records of the checking metrological reports, as well as other section-specific documents, such as: technical procedures, a software register, an equipment register, staff training records, correspondence, complaints and feedback, etc. 

A document that describes the structure and operation of this file system will be available. A master document list shall be maintained. This list shall refer to, as a minimum, the following items: 
\begin{itemize}
\item A register of Technical Procedures. 
\item A register of files other than commercial job files. 
\item A register of software used. 
\end{itemize}
Each item in the master document list shall identify:
\begin{itemize}
\item The document (name) 
\item The document location(s) 
\item The responsibility for maintaining the document
\end{itemize}

\subsubsection{Technical procedure files}
Sections have a number of technical procedures that describe specific technical activities ( see \S\ref{ss:technical_procedures} ). These procedures and documents related to them are filed by the section.
 
Note, the validity status of all technical procedures is recorded in the Validated Procedure Register in the central file system. The report prepared by the validation team when a procedure is validated is stored in the Technical Procedures Records library.

\subsubsection{Technical records}
Sections shall retain records of original measurements and derived data. The integrity of data and information in these records will be maintained; they shall be protected against tampering and loss.

Technical records provide a traceable link between the item under calibration and the issued calibration report. The records shall include sufficient information to enable identification of the factors contributing to the measurement uncertainty obtained and, if possible, allow the calibration to be repeated under equivalent conditions. The records shall identify the individuals who carried out the calibration and checked the results. 

Amended technical records shall be traceable back to previous or original versions. Original and amended data files shall be retained and a record kept of the changes made, the personnel responsible and the date of changes.
Technical records shall be retained indefinitely.

\subsubsection{Commercial jobs}
Documents related to the contractual arrangements for commercial jobs are managed in the ‘Job Register', a sharepoint smart-folder on the ‘Measurement Standards Laboratory' site. 

A section may use the ‘Job Register' as an extension of the section file system. 
For short jobs commissioned by IANZ, there is also an ‘IANZ Additional Jobs' library on the ‘Measurement Standards Laboratory' EDI site, which serves as an alternative to the Job Register. The procedure for using this register is written on the associated EDI page. 

\subsubsection{Software register}
Each section maintains the software used to carry out technical procedures. 
A software register will uniquely identify the software used with each technical procedure (including version numbers and its location). 

\subsubsection{Staff training records}
\label{sss:training_records}
A Training Record file will be maintained for each Member of the section, including contracted personnel. The file will record professional development activities and events, including relevant authorisations, approval of technical competencies (worker, checker, etc), educational and professional qualifications, training, skills and other experience. 
Training records must be available during audits and assessments. 

\subsubsection{Equipment registers}
\label{sss:equipment_register}
Each section will maintain an equipment register. The information in the register shall include: 
\begin{itemize}
\item Description 
\item Company asset register number 
\item Manufacturer/supplier 
\item Model number 
\item Serial number 
\item Location
\item Purchase date 
\item Calibration history 
\item Maintenance history
\end{itemize}
Other information may be included, as appropriate, such as:
\begin{itemize}\item Service agent 
\item Warranty expiry date 
\item Usage and personnel restrictions
\item References and manuals
\item Accessories
\end{itemize}

\subsubsection{Correspondence, complaints and feedback}
Correspondence may be filed at the discretion of the recipient in MSL. The appropriate location for filing the correspondence is also discretionary. 
A record of all complaints should be maintained. Complaints must be handled by raising an Improvement Request ( see~\S\ref{sss:irf_procedure}~).

Sections should keep records of positive feedback on work and services provided as well as client suggestions for improvement.

\subsection{Technical Procedures}
\label{ss:technical_procedures}
Sections use technical procedure documents to describe specific activities.

Technical procedures primarily describe how to carry out technical tasks. They are written for the person who will operate the procedure, which could be the author in several years time, or a colleague familiar only in general terms with the area. A procedure should contain sufficient information to enable a colleague to carry out the procedure in the absence of the author or include references to such information. It should also contain instructions on how to ensure confidence in the integrity of equipment used or contain references to such instructions.

Technical procedures need to be validated. They should contain a concise description of the relevant measurement science, sufficient for the purposes of reviewing the procedure for validation. References to relevant technical documents may be used if they are readily available. 

Technical procedures are validated for a definite time period (no more than 5 years). The validation is on the authority of the Quality Manager, based the recommendation of a validation team. 

\subsubsection{Technical procedure structure}
\label{sss:technical_procedure_structure}
A technical procedure document will contain the following sections:
\begin{itemize}
	\item Title, Author(s), Date
	\item Change history 
	\item Purpose and Description
	\item Health and Safety. This section will highlight Health and Safety issues related to operation of the procedure, e.g.: chemicals, cryogenic, electrical, radiation hazards, etc. It should reference any other documentation relating to the procedure or the equipment used, such as SOPs or MSDSs. 
	
	If the technical procedure is also functioning as an SOP then this section shall identify any hazards associated with operating the procedure, the risks posed by those hazards, and the controls for the risks. If controls (e.g. use of PPE) are associated with specific tasks or steps within the procedure then these should also be indicated at the appropriate places in the procedure description. 
	
	\textbf{Note, SOPs must be reviewed annually.} So, this requirement shall be planned for by the section independently of the validation requirements of the technical procedure. It may also be helpful to clearly identify technical procedures that contain SOPs. 
	
	\item Environment and equipment (information about the equipment and environmental requirements)
	\item If the procedure may be used outside MSL labs this needs to be stated, the limitations on the off-site environment need to be documented and any additional requirements for off-site work, including health \& safety precautions, need to be indicated
	\item CMC support (information about the IANZ classes and the BIPM CMCs that apply to results obtained, and the corresponding best uncertainties)
	\item References and Records (to papers, reports, manuals, files, computer programmes and files, laboratory books, and any other material that is needed to establish validation)
	\item Supporting procedures (this may be a standalone section or a subsection of References and Records). When critical traceability (i.e. directly determining the measurand) is derived from another procedure then that relationship must be acknowledged. Any procedure that is used to produce a value (even if that value is part of a traceability chain and will not be reported) must be in validation when used, or else when a report is issued. A dependency diagram may be helpful to explain how critical traceability is derived.
	
\end{itemize}

Note, any supporting procedure must be in validation when used to generate a value that is part of a traceability chain, unless a calibration report is issued directly to report the result, in which case the procedure should be valid when the report is issued. 

\begin{itemize}
\item Action (a description of what must be done, which may be in the form of a convenient guide or check list for the person carrying out the procedure) Note: controls (e.g. use of PPE) associated with specific tasks or steps may be required, see Health and Safety above.
\item Validation (any other information required to validate the procedure, such as measurement traceability and uncertainty calculations)
\end{itemize}

Other section headings may be used if appropriate.

A technical procedure may describe quality-assurance activities that provide evidence of satisfactory performance of the measurement capabilities used.  This may be needed to comply with MSL's quality assurance policy ( see \S\ref{ss:quality_assurance_policy} ).

\subsubsection{Amendments to technical procedures}
Amendments to a technical procedure can be made at any time. 

In general, an amended procedure must be re-validated. However, minor amendments can be made during the period of validity without triggering a full revalidation, provided that the date of the change is noted, as well as the people who made the change and reviewed it. These annotations must be made on the authoritative version of the procedure and incorporated in the procedure at the next validation.

A minor amendment is one that does not significantly change the method or its implementation, such as some additional explanation or detail.  When the authoritative version of the technical procedure is printed, handwritten amendments may be made.  When the authoritative version of the technical procedure is an electronic file, minor amendments may be made to the electronic file, as long as the procedure for making these amendments is documented.

Changes might be made to one part of a procedure that do not impact on the validity of other parts. In such cases, the unaffected parts may continue to be used without re-validation. Nevertheless, the claim that parts remain valid must be reviewed, and the reasons for accepting the claim noted on the authoritative version of the procedure.

\subsubsection{Technical procedure validation (and revalidation)}
\label{sss:tp_validation}
\begin{itemize}
\item A validation team, which must include at least one person with ‘reviewer’ competency for the procedure, will review a procedure and report on its validity.  The author of the procedure may not be a member of this team.
\item The team will check that: 
\begin{itemize}
\item the procedure is suitable for its intended purpose
\item evidence is provided that the physical process is sound
\item health and safety issues have been considered
\item evidence of measurement traceability has been provided
\item the process is described in enough detail to be carried out efficiently at a later date
\item the uncertainty analysis is complete and adequately documented (Note, the GUM approach to uncertainty analysis is based on the notion of a measurement model. It is strongly recommended that technical procedures provide such a model)
\item there is sufficient evidence to support the uncertainty claimed, such as
\begin{itemize}
\item an example of the least measurement uncertainty
\item results from a measurement comparison
\end{itemize}
\item references to MSL’s IANZ scope of accreditation and entries in the BIPM CMC database should be included as appropriate
\item references and records are readily available 
\item the requirements for data processing software have been adequately specified and that the software implementation has been validated
\item there are system and performance checks to provide on-going quality assurance in keeping with MSL's quality assurance policy ( see~\S\ref{ss:quality_assurance_policy} )
\item there is a list of things to watch out for (easily made mistakes, misunderstandings, etc), with comments as necessary
\item there is a suggested re-validation interval (no more than 5 years)
\item all staff identified in the TCM in respect to the procedure (i.e., author, worker, etc) have maintained the relevant competency
\end{itemize}
\item If any of the points above are found to be unsatisfactory the author will be asked to make suitable changes.
\item When re-validating a procedure, the validation team should ensure that material remains current (e.g., references, diagrams, etc)
\item The team will complete a Validation Cover Sheet (available in the Quality Manual library on EDI) and submit this, and the procedure, to the Quality Manager. 
\item The review team will ensure that a record of their analysis and notes about the procedure review are filed with technical procedure documents in the section’s file system. 
\item The Quality Manager will sign the Validation Cover Sheet to accept the validation. Alternatively, the Quality Manager may direct the validation team to reconsider part of their review.
\item The Quality Manager will ensure that information about a procedure is updated in the Validated Procedure Register ( see~\S\ref{sss:validated_procedure_register} ) and that a copy of the cover sheet and change history is saved in the Technical Procedure Records library in the central files.
\end{itemize}

When a technical procedure is validated, a new version number is allocated. 

Archival copies of all superseded technical procedures shall be retained by the section indefinitely.

\subsubsection{Validated Procedure Register}
\label{sss:validated_procedure_register}
A Validated Procedure Register keeps track of the status of all MSL technical procedures. Entries in this register identify the procedure, the version number and the period of validity.

The Validated Procedure Register is a sharepoint list in the central file system ( see \S\ref{ss:central_file_system} ). 

\paragraph{Procedure to modify the Validated Procedure Register}
\begin{itemize}
\item Following the review of a procedure by a validation team, a summary of the review will be presented to the Quality Manager, in the form of a validation cover sheet and a complete change history, as well as a copy of the procedure document.
\item The Quality Manager will sign the cover sheet if satisfied with the review.
\item One copy of the cover sheet will be filed, by the section, with the technical procedure. 
\item Scanned copies of the cover sheet and change history will be uploaded to the Technical Procedure Records library on EDI.
\item The corresponding entry in the register will be updated (by the site administrator) and checked by the Quality Manager (who sets the ‘approve’ attribute of the item – this change is recorded by the EDI versioning system).
\end{itemize} 

\subsubsection{Documentary standards}
Sections will ensure that copies of documentary standards required to carry out technical procedures, or referenced in technical procedures, are up to date. 
In general, the Callaghan Innovation Library should be consulted on how best to check for currency.

For instance, British Standards may be borrowed on a long-term basis from the Library, which will ensure automatic notification of updates. This includes any ISO or IEC standards that have been adopted in the UK by BSI. However, for other types of documentary standards (e.g. ASTMs, API, AS/NZS, AS) there is no updating/alerting in place. 
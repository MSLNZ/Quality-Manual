\section{Organisational structure}
\label{s:organisation}
MSL operates as a group within Callaghan Innovation. The structure of Callaghan Innovation and the structure of MSL are summarised below.

\subsection{Callaghan Innovation}
Callaghan Innovation is a Crown Agent of the New Zealand government. 

Callaghan Innovation is led by a Chief Executive Officer and divided into a number of business units, each managed by a General Manager reporting to the Chief Executive. Commercial Business is one of these units, within which MSL is an operational group. A Group Manager manages MSL. 

Inside groups, the role of Team Manager is recognised as someone who reports to the Group Manager and supervises staff. 

Callaghan Innovation also maintains a register of designated Work Area Managers. The role of Work Area Manager is important to the organisation in meeting statutory Health and Safety requirements. 
Work Area Managers report on Health and Safety issues to their respective Group Managers.

\subsection{Measurement Standards Laboratory}
MSL identifies the following roles:
\begin{itemize}
\item	Director
\item	Chief Metrologist
\item	Quality Manager
\item	Team Manager
\item	Member
\item	Quality Council Member
\end{itemize}
The appointment of staff in relation to MSL's role as New Zealand's National Metrology Institute (NMI) is described in section 2.2.3 below.
\subsubsection{Responsibilities} 
\begin{itemize}
\item	All staff in the MSL group at Callaghan Innovation and any other individuals authorised to carry out test and calibration work (i.e., individuals named in the Technical Competency Matrix) operate under the terms of this Quality Manual.
\item	The Chief Metrologist acts as a verifying authority in relation to the verification of any standard or standards of measurement in New Zealand.  The role of the Chief Metrologist shall be given to the Quality Manager or to another MSL Member.  
\item	The role of Director is of particular relevance to the Metre Convention. The Director also makes the necessary delegations to implement the QMS. The MSL Group Manager is currently the Director of MSL. 
\item	The Quality Manager ensures that the QMS is implemented and followed, oversees the management and improvement of the QMS and ensures that MSL's scope of accreditation, and its Calibration and Measurement Capabilities (CMCs) in Appendix C of the MRA, are 
consistent with each other.
\item	Members of the Quality Council (MQC) maintain the QMS, and the Quality Manual.

\end{itemize}

\paragraph{Impartiality}
\label{sssp:impartiality}
All staff shall act impartially.

Teams shall be responsible for identifying and documenting potential risks to impartiality. Team managers will take account of these records when assigning tasks to staff.

Potential risks to impartiality may be identified by the process described in the corporate document ``Declaring Conflicts of Interest and Managing Policy.docx''. Individuals' declarations will not be available to other staff, but team Managers will have access to this information and can manage the associated risks. The accreditation body will not be able to access this private information. 

The Risks and Opportunities register will be used to record potential risks to impartiality that are not subject to privacy constraints. The Risks and Opportunities register is part of the central file system (~see~\ref{ss:central_file_system}~).

\paragraph{The Quality Manager}
A Quality Manager is required by ISO 17025. 

Although the title Quality Manager need not be used, someone must be 
assigned the Quality Manager functions. 

A senior metrologist is usually appointed as Quality Manager. So, that person occupies a line management position under Group and Team managers. This should not present a problem provided that, on matters of quality, the Quality Manager has direct access to the highest level of management in the organisation at which policy and resourcing decisions are made affecting the 
laboratory.

The Quality Manager is responsible for implementing and operating the quality system on a day-to-day basis. He or she is normally responsible for administering the quality management system, for compiling the quality manual and for organising reviews and audits of the quality system.

While the role has a large administrative component, the Quality Manager is ultimately responsible for effective enforcement of quality policy. 

The Quality Manager is distinct from Group and Team managers. The responsibility is discharged through monitoring and advising not by managing technical activities directly.

The Quality Manager will advise management on quality issues and also has a responsibility to report to the APMP Technical Committee on Quality Systems on matters of quality relating to MSL.
The Quality Manager role requires familiarity with ISO 17025 and knowledge of the MSL quality system. Suitable experience could be acquired, for example, by acting as an IANZ Technical Expert to assess commercial calibration laboratories, and by serving on the MSL Quality Council.

\subsubsection{MSL Sections}
MSL is organised into three technical teams, each comprised of sections aligned with different technical areas of the SI (~see figure~\ref{fig:org_msl}~).

The sections are: 
\begin{itemize}
\item	Length
\item	Time and Frequency 
\item	Electrical 
\item	Temperature and Humidity
\item	Photometry and Radiometry
\item	Mass and Pressure
\end{itemize}

Sections are involved in:
\begin{itemize}
\item	maintenance and development of measurement standards; 
\item	calibration and measurement services;
\item	scientific research;
\item	liaison with international metrology groups and networks of peers; 
\item	dissemination of knowledge and technical services within New Zealand.
\end{itemize}

Third-party accreditation of MSL by IANZ is organised by section, and sometimes by sub-section (e.g., Pressure, Mass and related quantities, Temperature, and Humidity are typically treated as distinct disciplines for technical reviews).

\begin{figure}
\begin{center}
\includegraphics[width=1.1\textwidth]{pictures/MSL_org_full}
\end{center}
\caption{MSL Organisational Structure (September 2, 2022)}
\label{fig:org_msl}
\end{figure}


\subsubsection{Approvals}
\paragraph{Chief Metrologist}    

The appointment of the Chief Metrologist in relation to MSL's role as the New Zealand NMI is done in the context of the Measurement Standards Act 1992 \cite{MS_ACT_1992} and the National Standards Regulations 2019 \cite{NS_Regulations}.
The National Standards Regulations 2019 (as at 20 May 2019) has the definition:
\begin{quote}
Measurement Standards Laboratory of New Zealand means the laboratory of Callaghan 
Innovation (established under section 7 of the Callaghan Innovation Act 2012) that maintains 
the principal standard measures for New Zealand.
\end{quote}
This is interpreted as being the responsibility of Callaghan Innovation to define (effectively delegated from the Minister, through the regulation). Under this interpretation, the Chief Executive Officer shall appoint, in writing, the Chief Metrologist. The Chief Metrologist shall be a senior metrologist with knowledge of and familiarity with the New Zealand and international measurement infrastructure. The roles of Director and Chief Metrologist may be combined.
Under the National Standards Regulations, the Chief Metrologist may delegate, in writing, to any other person who is an MSL Member.
\paragraph{Team Manager}
Technical activities are organised into three teams, each with a manger who has overall responsibility for technical operations, including provision of the resources needed to ensure the required quality of laboratory operations. 
\paragraph{Quality Manager}
The Director appoints the Quality Manager. Appointments are for a period of two years, which may be extended. 

The Quality Manager has the responsibility and authority to ensure that the quality management system is implemented and adhered to at all times; the Quality Manager has direct access to the highest levels of management at which decisions are made on laboratory policy or resources (currently the General Manager Commercial Business).

The Quality Manager may delegate, in writing, to any MSL Member.

When the Quality Manager is absent, the Chief Metrologist automatically assumes the relevant responsibilities.  The Chief Metrologist may re-delegate such responsibilities. 
%
\paragraph{Member}
Members are qualified for particular roles in relation to specific technical procedures. Members are listed in the Technical Competency Matrix (TCM, the TCM is part of the central files system described in section 4.2 ). Note, some individuals in the TCM may be under contract to MSL or be employees of Callaghan Innovation outside MSL. 

An MSL Member who has been approved as a signatory by IANZ for particular items in the scope of accreditation is authorised to sign IANZ endorsed certificates and reports for these items.

%\deprecated{Note that in clause 8(2)(a) of the National Standards Regulations 1976 (as at 1 May 2014), with Amendment No.1 (1992), which relates to delegation of the powers of a verifying authority, the reference to 'any other person who works in that laboratory;' is interpreted as an MSL Member. }

{Note that in clause 8(2)(a) of the National Standards Regulations 2019, in relation to delegation of the power of a verifying authority, the reference to 'a person working in the Measurement Standards Laboratory of New Zealand' is interpreted as an MLS Member.}

\paragraph{Quality Council}
After consultation with the Director, the Quality Manager appoints, in writing, the members of the MSL Quality Council (MQC). Membership tends to be for periods of several months, so that representation can be shared across MSL over time.

\paragraph{Leadership Team}
The MSL Leadership Team includes the Director, the Chief Metrologist, the Quality Manager, the 
Team Managers and the Science Support Coordinator. 

\subsubsection{MSL Affiliations}
\label{sss:affiliations}
\begin{itemize}

\item	MSL is accredited, by IANZ, as an accredited Metrology and Calibration Laboratory complying with NZS/ISO/IEC 17025 (original date of accreditation, 30 July 2004). 

\item	The Chief Metrologist is authorised by the Minister of Science and Innovation, under the Measurement Standards Regulations, 
%\deprecated{1976, Amendment 1}%
to act as a verifying authority in relation to the verification of any standard or standards of measurement in New Zealand.

\item	New Zealand is a signatory to the Metric Treaty of 1875 and hence is a member state of the Metre Convention.  New Zealand (through the Ministry of Business, Innovation and Employment) subscribes annually to the Comit\'e International des Poids et Mesures (CIPM) which administers the Metre Convention, provides technical coordination, and is the owner of the Syst\`eme International d'Unit\'es (SI). The Ministry of Business, Innovation and Employment handles the administrative aspects of New Zealand relationship with the CIPM, 
while MSL handles the technical aspects, and provides a New Zealand representative (normally the MSL Director) to the Conf\'erence G\'en\'erale des Poids et Mesures (CGPM).

\item	MSL is a signatory to the Mutual Recognition Arrangement on National Measurement Standards and Calibration Measurement Certificates Issued by National Metrology Institutes (MRA) drawn up by the CIPM under the authority given to it under the Metric Treaty.

\item	MSL is a permanent member of four of the Consultative Committees (CCs) that assist the BIPM with its scientific work: CCT (Thermometry), CCEM (Electricity and Magnetism), CCPR (Photometry and Radiometry) and CCM (Mass and related units). 

\item	MSL and IANZ have signed a Memorandum of Understanding (MOU) formalising the relationship between the national metrology institute and the national laboratory registration agency, as required for recognition by the European Union (EU), of New Zealand testing and certification. The Measurement and Calibration Professional Advisory Committee, chaired by the Chief Metrologist, provides the vehicle for operating the MOU.

\item	MSL and the National Measurement Institute, Australia (NMIA), have signed seven agreements recognising the equivalence of the following units: metre, kilogram, volt, ohm, farad, second, kelvin.

\item	MSL is a full member of the Asia-Pacific Metrology Programme (APMP). This Regional Metrology Organisation (RMO) is a member of the Joint Committee of Regional Metrology Bodies and the BIPM (JCRB) that plays a key role in operating the MRA.
\end{itemize} 

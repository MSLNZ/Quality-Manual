\section{Review of requests and contracts}
\label{s:requests_and_tenders}
\subsection{Policy}
This policy covers calibration, testing and consultancy work for clients inside and outside the organisation.

Any new request shall be reviewed to determine whether the work requested is adequately defined, whether MSL has an appropriate capability and sufficient resources to undertake the work, and whether the service can be supplied in an acceptable timeframe.

A record of the review will be kept with other job records in the section file system.

As a provider of testing services, it is important to understand the nature of the service that a client is requesting. If there are any differences between the request and the work that will be carried out at MSL, including the timeframe, these differences must be resolved before a contract is agreed.

\subsubsection{Confidentiality}
\label{sss:confidentiality}
MSL must keep confidential all information obtained from clients, and we must inform the client of this responsibility. The organisation’s Work Order Agreement clearly states this, and, in most cases, a Work Order Agreement will be signed by the client. 

However, when a Work Order Agreement is not used, the client must be made aware of MSL’s responsibility to keep information confidential. In such cases (expected to be jobs of very low monetary value), the following text may be adapted to the particular work and given to the client (e.g., by email):
\begin{quote}
\textit{Further to our agreement that Callaghan Innovation will conduct work for you, being to [insert] (Work), please note that we will maintain as confidential at all times all information that you share with us in any form, except as reasonably required for the purpose of the Work or where we are required by law or governmental authority, or where authorised by you.}
\end{quote}


\subsubsection{Additional work for IANZ}
IANZ engages MSL for ad hoc services from time to time. The nature of these jobs is varied. Examples include: a request for a technical guidance document about a specific type of measurement; review of existing documents; the technical evaluation of documents supporting a scope extension by one of IANZ’s clients; review of software tools. 

The review of a new request from IANZ is also covered by this policy.

There is a streamlined process for registering and managing IANZ work, which available as an alternative to the usual job registration process. If this streamlined process is used, the record of review should be filed in the generic job for that section’s IANZ work.
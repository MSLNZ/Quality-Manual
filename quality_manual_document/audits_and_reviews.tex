\section{Audits and reviews}
\subsection{Internal Audits}
\label{ss:internal_audit}
A comprehensive internal audit of MSL technical sections and of the quality management system must be carried out annually. 

It is the responsibility of the Quality Manager to plan and organize internal audits, to appoint audit teams, and to ensure that actions in response to the findings of the audits are carried out. Prior to the audits, the Quality Manager shall hold a meeting with the audit teams and Team Managers in order to set expectations around the process, highlight any particular areas of focus or answer any other questions related to the audit. The audits will be held at a time convenient to both the audit team and the section members. 

Some sections are sub-divided into smaller technical areas (e.g., Photometry and Radiometry, Mass and Pressure, Temperature and Humidity). Audits of these areas alternate, so each is audited biennially.

The QMS audit complements the technical section audits by reviewing overarching aspects of the quality management system. The review should look at those parts of the QMS that have not been delegated to technical sections, such as the central filing system, improvements, meetings, etc. A subset of clauses from the standard will be reviewed each year (MSL QS form). When selecting clauses, consideration should be given to those already covered in recent audits. 

For all audits, a summary of corrective actions and recommendations from previous audits, as well as any IRFs associated with the section, should be reviewed by audit teams before conducting an audit: a response should be recorded against each recommendation arising from earlier audits. A record of previous internal audit findings is kept in the Internal Audit Register in the MQC Documents library of the central file system. A summary of corrective actions and recommendations from previous IANZ audits is kept in the IANZ Audit Register, also in the MQC Documents library of the central file system. 

The audit team shall identify any quality system non-compliance by requesting a corrective action (CAR). The audit team may also recommend improvements (R) and make suggestions (S). 

Within one week of the audit, the audit team shall produce a report on the scope, findings and decided actions in response to the audit, and provide a copy of that report to the Team Manager and section members involved in the audit.  Corrective actions, recommendations and suggestions shall be clearly listed in the audit report (e.g. C1, R1, S1 etc). An Internal Audit Check-list may be used to assist documentation of the audit process for technical sections. It is crucial the content of the audit report has been reviewed and agreed between the audit team, section members and the Team Manager prior to going through the sign off process.

The report shall first be signed by at least one member of the audit team.  The Team Manager shall then sign the report to indicate acceptance of the audit findings (if areas of disagreement cannot be resolved, they may be noted on the signed document), ensuring the document has been reviewed by the section members before doing so. Corrective actions arising from the audit shall be raised immediately as improvement requests (IRFs) by the Team Manager.

The report will be provided to the Quality Manager no later than two weeks after the audit. The Quality Manager will sign the report when satisfied (and after ensuring that any IRFs have been raised). 

A copy of the report will be filed in the Internal Audits library of the central file system. The Internal Audit Register file in the MQC Documents library will be updated by the Quality Manager.

\subsection{Management review}
\label{ss:management_review}
A management review of MSL will be carried out annually by the laboratory’s top management. 

The purpose of the review is to assess the suitability and effectiveness of the laboratory's management system, calibration and testing activities, and to introduce any necessary changes or improvements.  

The Management Review Panel Guidelines \cite{MSL_Management_Review_Guidelines} contains operational expectations for the running of the review.

\subsubsection{Management review procedure}
\begin{itemize}
\item The MSL Director will organise a management review once every twelve months. 
\item The review team shall consist of, at least, the Chief Innovation Expertise Officer, an MSL Leadership Team member, and a member of the MQC.
\item The review is intended to assess the suitability and effectiveness of the MQS and shall consider, at least:
\begin{itemize}
\item The previous management review report;
\item Reports from the Quality Manager, the Chief Metrologist and the Team Managers; 
\item Submissions from staff
\end{itemize}
\item The Quality Manager shall report on the effectiveness of the quality system since the previous management review. In particular:
\begin{itemize}
\item Status of quality objectives;
\item Status of external and internal audits;
\item Status of improvement requests, risks and opportunities;
\item Activities of the Measurement Quality Council;
\item Assessments of customer satisfaction;
\item Suitability of policies and procedures;
\item Other factors, such as monitoring activities and staff training.
\end{itemize}
\item The Chief Metrologist shall report on the effectiveness of MSL as New Zealand’s NMI. In particular:
\begin{itemize}
\item The suitability and effectiveness of technical competencies and technical section measurement capabilities;
\item MSL’s participation in international measurement comparisons;
\item MSL’s scientific activities and outputs.
\end{itemize}
\item Team Managers shall report on the effectiveness of the technical sections in their teams. In particular: 
\begin{itemize}
\item Changes in the volume and type of work or in the range of activities;
\item Resources, staffing and technical competencies;
\item Improvement requests, risks and opportunities;
\item Outcomes of external and internal audits;
\item Traceability, standards maintenance, development and quality assurance;
\item Scientific activity and output.
\end{itemize}
(Note, where bullet points under Team Manager reporting coincide with items under the Quality Manager or Chief Metrologist reporting, it is intended that Team Managers report on specific details for their technical sections.) 
\item The review team shall produce a report on any decisions and actions in relation to, at least:
\begin{itemize}
\item The effectiveness of the management system and its processes;
\item Improvements to laboratory;
\item Provision of required resources;
\item Any needs for change.
\end{itemize}
\item The report will be filed in the MSL Main AO Docs library. Management shall ensure that actions are carried out within an appropriate and agreed time frame.
\end{itemize}

Note that quality system issues are discussed at monthly MSL Management and Quality meetings attended by the Quality Manager, the Chief Metrologist, the Team Managers and the Director. This is considered another part of the management review process.
\subsubsection{Management review reporting}
The Quality Manager will ensure that MSL staff are made aware of the new Quality Objectives for the year.
\section{Purpose}
The main purpose of the Quality Management Review is covered in the MSL Quality Manual \cite[\S\ref*{QM-ss:management_review}]{MSL_Quality_Manual}: 
\textit{``to assess the suitability and effectiveness of the laboratory's management
system, calibration and testing activities, and to introduce any necessary changes or improvements''}. The review will also assess effectiveness in achieving the wider goals of MSL as a
national metrology institute.

\section{Terms of reference}
The quality management review is an opportunity to critically analyse the goals and objectives of the MSL quality system and to receive affirmation that current plans and priorities
will have long-term benefits for MSL.

The review is expected to address the following questions:
\begin{description}
\item[Current state: ]What is the current situation of MSL? What does MSL do well to achieve
its quality objectives? How does the laboratory contribute to the quality objectives, wider
goals and strategies of MSL?

\item[Future state: ]What challenges face the quality system and development of MSL? What
changes might be required to strengthen the laboratory's contribution to MSL's goals? What
can the parent organisation do to support the laboratory to achieve its goals and quality
objectives?
\end{description}


\section{Key participant roles and responsibilities}
\subsection{Review Organizer and Secretary}
\begin{description}
\item[Review organizer: ]The MSL director will assume this role and initiate the review process.

The organizer shall: 
\begin{itemize}
\item nominate a review secretary;
\item select the panel members and confirm
their availabilities;
\item confirm the date and time for the review;
\item make sure all required review
documents are provided in a timely manner and are made available to the panel and MSL
staff members before the review;
\item   make sure a panel review report is produced.
\end{itemize}  
\item[Review secretary: ] This role shall be nominated by the organiser. 

The secretary shall: 
\begin{itemize}
\item arrange the review meeting venue;
\item send invitations to the meeting to all MSL members;
\item call for review submissions;
\item distribute review documents to the panel and MSL staff members;
\item take minutes during the meeting;
\item  provide any other administrative support as required by the organiser and the panel.
\end{itemize}   
\end{description} 

\subsection{The Panel}
The panel should consist of at least 3 members: the MSL director, the CI executive member
overseeing MSL and an MQC member.

Before the review meeting: consider reports from previous management reviews, reports
from the quality manager, chief metrologist and team managers as well as any submissions
from staff; familiarise themselves with the MSL quality manual and all related documents
including quality objectives, strategy documents, improvement requests and risk \& opportunities register.

If required, the MQC panel member may provide advice to the panel on the MSL quality
system and clarify issues and concerns specific to MSL. The panel is encouraged to meet
prior to the review to set the focus of the review, discuss any potential issues and decide on
the review meeting agenda.

Following the review meeting, the panel should collectively produce a panel review report.

\section{Self-review reports and submissions}
Self-review reports form the basis of submissions to the panel. The quality manager, chief
metrologist and the MSL team managers are required to provide these reports in a timely
manner before the review meeting. Scope of the reports are covered in MSL Quality Manual
\cite[\S\ref*{QM-ss:management_review}]{MSL_Quality_Manual}.

MSL staff members may give submissions to the panel regarding the effectiveness of the
quality system. The panel shall consider all of the submissions.

\section{Review meeting}
All MSL staff members shall be invited to the review meeting. The review organiser will run
the meeting according to the agenda and facilitate any discussions. The panel shall ask
questions prompted by the self-review reports and lead discussions on any staff submissions with the meeting. The panel may put forward recommendations and implementation
plans for discussion.

Reports are considered as read---5 minutes overview per reviewee can be allocated. Time
shall be planned for submission writers to explain their submissions and provide more context. Time should also be allowed for attendees to ask questions and comment on the
reports and submissions.

\section{Panel review report}
The panel is required to produce a report at the conclusion of the review. The report should
include the recommendations and implementation plans that were ideally discussed and
agreed upon during the meeting. This is covered in the quality manual \cite[\S\ref*{QM-ss:management_review}]{MSL_Quality_Manual}: any decisions and actions in relation to, at least: 
\begin{itemize}
\item  the effectiveness of the management system and its processes; 
\item improvements to laboratory; 
\item provision of required resources; 
\item any needs for change.
\end{itemize} 
This report will be published to all MSL members and filed in the quality management system
repository.